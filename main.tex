\documentclass[defaultstyle,11pt,inlineh4]{thesis}
% \documentclass[11pt,openany,oneside]{book}

\usepackage{amssymb}		% to get all AMS symbols
\usepackage{amsmath}
\usepackage{graphicx}		% to insert figures
\usepackage{hyperref}		% PDF hyperreferences??
\usepackage{todonotes}
\usepackage{comment}
\usepackage{booktabs} 
\usepackage{bbm}
\usepackage{tabularx}
\usepackage{romannum}

%% Math Macros
\newcommand{\reals}{\ensuremath{\mathbb{R}}}
\newcommand{\users}{\mathcal{U}}
\newcommand{\items}{\mathcal{V}}
\newcommand{\features}{\mathcal{F}}
\newcommand{\domain}{\mathcal{D}}
\newcommand{\pro}{\mathcal{P}}
\newcommand{\sensitive}{\mathcal{S}}
\newcommand{\rerankers}{\mathcal{K}}
\newcommand{\metrics}{\mathcal{M}}

% \setcounter{tocdepth}{1}


\newtheorem{definition}{Definition}
\newcommand{\libauto}{\texttt{librec-auto}}

\usepackage{caption}
\usepackage{subcaption}
\usepackage{color}
\usepackage{colortbl}
\usepackage{multirow}
\usepackage{enumitem}
\usepackage{algorithm}
\usepackage{algorithmic}
\usepackage{enumerate}
\usepackage{multirow}
\usepackage{float}
\usepackage{mathrsfs} % needed for mathscr
\usepackage{color}
\usepackage{balance}
\usepackage[bottom]{footmisc}
\renewcommand{\algorithmicrequire}{\textbf{Input:}}
\renewcommand{\algorithmicensure}{\textbf{Output:}}
\usepackage[para,online,flushleft]{threeparttable}
\usepackage{verbatim}
\usepackage{dsfont}
\usepackage{tikz}
\usetikzlibrary{shapes,arrows,spy,positioning,decorations,decorations.pathreplacing}






%%%%%%%%%%%%   All the preamble material:   %%%%%%%%%%%%

% \newcommand{\dtitle}{Controlling the Fairness Accuracy Tradeoff in Recommender Systems}
% \begin{document}

\title{Controlling the Fairness / Accuracy Tradeoff in Recommender Systems}


\author{Nasim}{Sonboli}

\otherdegrees{B.A. Computer Engineering-Software, Islamic Azad University, Karaj Branch, 2011 \\
	      M.S. Data Science, DePaul University, 2016}

\degree{Doctor of Philosophy}		%  #1 {long descr.}
	{Ph.D., Information Science}		%  #2 {short descr.}

\dept{Department of}			%  #1 {designation}
	{Information Science}		%  #2 {name}

\advisor{Prof.}				%  #1 {title}
	{Robin D. Burke}			%  #2 {name}

\reader{Bamshad Mobasher}		%  2nd person to sign thesis
\readerThree{Bo Waggoner}		%  3rd person to sign thesis
\readerFour{Casey Fiesler}      %  4rd person to sign thesis
\readerFive{Michael Ekstrand}   %  5rd person to sign thesis


\abstract{  \OnePageChapter	% because it is very short, Leave out the \OnePageChapter if your abstract runs over one page.

	Often the abstract will be long enough to require
	more than one page, in which case the macro
	``$\backslash$OnePageChapter'' should {\it not}
	be used.

	But this one isn't, so it should.
	}

\dedication[Dedication]{	% NEVER use \OnePageChapter here.
	To all of my fluffy socks.
	}

\acknowledgements{	\OnePageChapter	% *MUST* BE ONLY ONE PAGE!
	Here's where you acknowledge folks who helped.
	But keep it short, i.e., no more than one page,
	as required by the Grad School Specifications.
	}

% \IRBprotocol{E927F29.001X}	% optional!

\ToCisShort	% use this only for 1-page Table of Contents

\LoFisShort	% use this only for 1-page Table of Figures
% \emptyLoF	% use this if there is no List of Figures

\LoTisShort	% use this only for 1-page Table of Tables
% \emptyLoT	% use this if there is no List of Tables

%%%%%%%%%%%%%%%%%%%%%%%%%%%%%%%%%%%%%%%%%%%%%%%%%%%%%%%%%%%%%%%%%
%%%%%%%%%%%%%%%       BEGIN DOCUMENT...         %%%%%%%%%%%%%%%%%
%%%%%%%%%%%%%%%%%%%%%%%%%%%%%%%%%%%%%%%%%%%%%%%%%%%%%%%%%%%%%%%%%


\begin{document}

\input macros.tex
% \input chapter1-intro.tex
% \input chapter3-fairness.tex
% \input chapter4-methodology.tex
% \input chapter5-inprocess.tex
\input chapter6-postprocess.tex
\input chapter7-librec.tex
% \input chapter8-conclusion.tex

%%%%%%%%%   then the Bibliography, if any   %%%%%%%%%
\bibliographystyle{plain}	% or "siam", or "alpha", etc.
\nocite{*}		% list all refs in database, cited or not
\bibliography{refs,refs_libauto}		% Bib database in "refs.bib"

%%%%%%%%%   then the Appendices, if any   %%%%%%%%%
% \appendix
% \input appendixA.tex
% \input appendixB.tex


\end{document}

