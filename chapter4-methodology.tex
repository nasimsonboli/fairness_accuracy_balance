\chapter{Experimental Methodology}
\label{ch:methodology}

Personalized recommendation is domain-specific; in other words algorithms that work in one setting might work poorly in others that might have different characteristics. \todo{\cite{}}
In this Chapter, I will describe the datasets, data pre-processing, data transformation, as well as recommendation and re-ranking algorithms that were used as baseline to perform experiments on them throughout this dissertation. 
Additionally, experimental details such as hyper-parameter tuning, the choice of the models and the choice of the protected attribute(s) in each dataset will be explained thoroughly. For my experiments, I have used and/or extended \libauto{}, a recommendation library that I worked on during the course of my PhD. The extended algorithms were added as part of my contribution to \libauto{} or LibRec itself. I will describe \libauto{}'s capabilities and architecture in details in Chapter \ref{librec-auto}.

\todo[inline]{what do we mean by protected and unprotected groups?}

\section{Context}
    \subsection{Philanthropic - loan recommendation}

    \todo[inline]{One important characteristic of this dataset, and the micro-finance domain in general, is there is rapid turn-over in recommendable items. Loans are only available to lenders for a short period until they are fully funded or dropped from the system. Subsequent visitors will not see or be able to support these loans, limiting the maximum item profile size significantly. For example, at a minimum loan amount of \$25, a \$200 loan can have a maximum of only 8 lenders. Contrasting with a consumer taste domain such as MovieLens \cite{movielens}, where a popular movie might be rated by hundreds or thousands of consumers, the Kiva.org dataset is extremely sparse (with a sparsity of $4.19\times 10^{-5}$) and exhibits a significant item cold-start problem.}
    % Microlending is the provision of small and low-interest loans (as little as \$25) to low-income individuals or small-scale entrepreneurs from under-developed countries \cite{yunus1998banker}. The extremely poor people in rural areas often lack collateral, steady employment, or a verifiable credit history, hence they cannot get access to financial services. Under such circumstances, microlending has attracted an increased attention in the last decade \cite{chen2017microfinance}, providing impoverished entrepreneurs an opportunity to start their own businesses as well as avoid the vicious cycle of debt.
    
    % One of the leading international microlending organizations, Kiva Microfunds (Kiva.org), has crowd-funded about 3 million borrowers with \$1.27 billion USD as of February 2019 \cite{kiva}. Kiva does not collect interest but provides an intermediary service for lenders and borrowers, as illustrated in Figure \ref{fig:kiva_process}. Borrowers from over 80 countries, divided into 8 regions by Kiva, post their applications for loans on the website for lenders to support. Lenders browse and crowd-fund the loans in the increments of \$25 or more. 
    
    
    Online sites have become a major avenue for people not only for shopping, but for philanthropic activities. Kiva Microloans is a non-profit organization that operates a crowd-sourced microlending platform. Kiva uses crowdsourcing to provide access to capital for individuals and entrepreneurial groups, especially in the developing world, who are otherwise under-capitalized.
    % with the goal of financial global inclusion. 
    
    Kiva partners with local organizations in countries across the globe and, since its founding, has lent \$1.46 billion to 3.4 million borrowers in 77 countries with the support of 1.9 million lenders. Information about lending opportunities are listed on Kiva's site for a fixed period and promoted to its lenders through various means. A typical lender will contribute only a fraction of the total amount (as low as \$25) for any one loan, but may support multiple loans at any one time. As with any lending transaction, there is risk that the borrower will not repay, but this is rare in Kiva where the repayment rate is approximately 96\%. Lenders do not get any interest on their investments and so supporting a Kiva borrower is essentially a philanthropic act.

    The main sides of the interaction, therefore, are the borrowers, generally developing-world entrepreneurs who seek small amounts of capital to enhance their business capacity, and lenders, who are the application's end-users. Note that the borrowers are the providers or the producers of the items/loans, and the lenders are the receivers or users of the recommendations. We will use these words interchangeably throughout this dissertation. Kiva's mission emphasizes equitable access to capital for its borrowers, who generally cannot make use of traditional forms of banking and lending~\cite{Choo_understanding_kiva}.
    
    
    A loan recommender system in this situation is intended to lower the search costs of its users (lenders) by finding borrowers whose goals and needs appeal to them. In this recommendation context, several provider-side fairness concerns arise, no consumer-side fairness has risen as a concern so far. 
    
    One key provider-side concern, arising from Kiva's mission of supporting world-wide access to capital, is that the geographic imbalances in users' preferences may manifest themselves in the disproportionate representation of certain countries or regions in recommendation lists. This could give rise to a positive feedback loop, as the recommended items are more likely to be supported, and thus the lending becomes even more highly concentrated. A similar kind of imbalance may arise with respect to different industries or economic sectors. Thus, we can identify at least two fairness concerns within this recommender system: equity in geographic distribution of capital \cite{liu2019personalized}, and equity across economic sectors \cite{sonboli2020opportunistic} which will be explained in detail in Chapter \ref{fairness_postproc}. Other provider-side fairness concerns might arise in this context, e.g., the loan amount (loans with higher amounts get funded slower). Although we have mainly addressed the first two aspects in this thesis.
    
    
    We situate our research within the organizational context of Kiva Microloans. This is a compelling case study for the exploration of recommendation fairness since its fairness requirements are driven by internal needs surrounding its philanthropic mission, rather than external demands such as regulatory requirements that might be found in financial services or employment, or etc. A regulatory environment is more likely to provoke a defensive response related to fairness questions, which tends to hamper robust discussion of fairness properties in existing systems~\cite{chen2018fair,holstein2019improving}. In addition, because of its philanthropic mission, it is reasonable to expect that Kiva's users will be receptive to fairness-oriented interventions. Finally, as a hybrid organization, embodying the characteristics of a nonprofit with the characteristics of a financial services institution, research embedded with Kiva is well-situated to have a broader and more generalizable impact across genres of institutions and sectors.

    % Loan recommendation, as in the case of Kiva is a suitable candidate for the exploration of recommendation fairness due to several reasons. One of them is that fairness requirements are driven by the internal needs surrounding Kiva's philanthropic mission to provide equitable access to capital for its borrowers regardless of their demographic information such as gender, the country of origin, etc., as well as other characteristics e.g. the economic sector for which they need the loan. Another reason reason on the suitability of Kiva is as follows: Kiva's users will be more receptive of the fairness-oriented interventions in recommendations due to its philanthropic nature whereas the users of a movie streaming website such as Netflix, might not care to be fair to the movie producers. 


    \subsection{Streaming media/E-commerce - movie recommendation}

    Streaming services for digital media consumption provide users with on-demand access to global content, thereby connecting users with content creators, e.g., users and movie makers on video streaming platforms like Netflix.
    
    Algorithmically generated recommendations power and shape the bulk of consumption patterns on such platforms. As \cite{mehrotra2018towards} note, to maximize user satisfaction, streaming platforms that are optimizing for relevance, could inadvertently be detrimental to exposure of the providers or content creators such as movie producers on the tail-end of the spectrum. Optimizing only for relevance not only poses risks of provider-unfairness, but it can also affect the recommendations of consumers themselves. As \cite{eskandanian2019power} notes, this phenomenon in collaborative filtering, can form a powerful minority of users (as low as 5\%) which can affect the recommendations of the majority of the users (as big as 95\%). This will cause a miscalibration in the recommendations of the users as it is discussed in \cite{steck2018calibrated} which can be considered as a type of consumer-side unfairness. 
    
    Therefore streaming platforms carefully consider the influence of their recommendations on consumption patterns in a manner benefiting not only the users, but also creators, and the long term goals of the platform itself. To better navigate such trade-offs between different stakeholder objectives, platforms are increasingly relying on multi-objective methods to jointly optimize multiple user-centric goals (e.g., engagement metrics like clicks, number of songs played, time spent), provider-centric goals (e.g., exposure) and platform-centric goals (e.g., fairness, discovery, diversity) \cite{mehrotra2020bandit}. In certain cases, aptly balancing such varied objectives makes it possible to obtain gains in both complementary and competing objectives.
    
    In this dissertation, we have have chose the movie recommendation context due to several reasons. Similar with the case of Kiva, fairness concerns regarding the exposure of the providers arises in these systems as well. We have addressed this issue in Chapter \ref{fairness_postproc} in \cite{sonboli2020opportunistic,liu2019farpfar} \todo[inline]{cite dynamic fairness project}.
    
    Movie recommendation might not be an obvious candidate for the study of consumer-side fairness in recommendation. Although, we followed the example of \cite{yao2017beyond} and used an approach to construct an artificial equity scenario within both MovieLens-1M and The Movies data for expository purposes only, with the understanding that real scenarios can be approached with a similar methodology. We chose MovieLens datasets due to the other following reasons: (a) MovieLens-1M is among the very few datasets with demographic information which makes it suitable for the use case of consumer-side fairness, (b) MovieLens-1M is a standard and commonly used dataset in the field of recommendation systems, and (c) generally lack of datasets in domains where fairness matters.

    Below we characterize in detail these datasets.
    
\section{Datasets}

The experiments were performed on the following publicly and/or proprietary datasets: MovieLens\cite{movielens}, The Movies dataset\footnote{https://www.kaggle.com/rounakbanik/the-movies-dataset}, and two variations of Kiva dataset: proprietary and publicly accessible data.
It's worth mentioning that an anonymized data was created based on the proprietary Kiva dataset which was used in my experimentation that we are allowed to share with the research community with Kiva's permission.
\todo[inline]{is it correct?}

 \todo[inline]{Yelp dataset. do we need that?}


The characteristics of the previously mentioned datasets are described below and summarized in Table \todo[inline]{create a table for them}. Below, we will describe the choice of the protected attribute(s) and the transformation methods used to produce them for each dataset.
% Dataset & #users, #items, #ratings, rating scale, density, consumer, provider, protected attribtue, access (public or private)

    \subsection{MovieLens 1M Dataset}
    For some of our experiments, we are using the well-known MovieLens 1M dataset~\cite{movielens}. This dataset is a movie rating dataset that was collected by the GroupLens research group. This dataset contains 1,000,209 ratings from 6,000 users on 4,000 movies. The sparsity of MovieLens is 4.47\%. All of the users in the MovieLens data set rated at least 20 movies.
    Besides user ratings, this dataset contains, some additional information on ratings, users and the movies. The rating information consist of ratings (only whole ratings), and the timestap of the ratings (in terms of seconds). The movie information consists of movie titles and 18 movie genre(s) (e.g. "Animation", "Comedy"). It also contains user information such as: gender (binary, represented by F and M), age (split into 7 buckets), 20 occupations (e.g. "academic/educator", "artist") and zip-code.
    
    
    \todo[inline]{check the next paragraph (copied from teh dynamic lottery project)}
     In this dataset, three types of genders were present: 0, 1, 2. And each movie can be directed or written by a group of directors or writers. To capture this diversity, gender was discretized into seven groups. For example if a movie is directed by all the genders, we assign 012 for the gender information and if it is directed only by one gender, a single number was assigned to that movie e.g. 0, 1 or 2. All the categorical features were transformed into dummy variables, resulting in a total of 335 binary features.
    
    \todo[inline]{check this next as well}
    In a fielded application, the choice of sensitive features and protected groups within those features may be determined by legal liability or business model considerations. Lacking this type of insight, we chose to identify protected features as those associated with rarely-recommended items. To determine the protected values of each feature, we performed a trial run of recommendation generation over the data set, and examined the distribution of features in the results. In a live system, historical recommendation data would be available over which to calculate this distribution. The values in the 25th percentile of the distribution were selected as the protected group for that feature.

    
        \todo[inline]{the slim-u discussion is out of place here!}
        \subsubsection{protected feature(s)}
        The choice of the protected features firstly depend on the stakeholder for which we are seeking fairness, and secondly it depends on the context and the previously detected unfairness or biases in the dataset. Since, there exists information about both the consumers (of the recommendations) and the providers (of the recommendations), it is possible to propose solutions both for the consumer-side and provider-side fairness.
        
        % balanced neighborhood
        In Chapter \todo[inline]{refer to balanced neighborhood}, we propose BLN-SLIM, an in-processing approach to mitigate both consumers' (BLN-SLIM-U) and providers' (BLN-SLIM-I) biases. Note that this approach takes a single binary protected attribute as input to the algorithm.
        
        Although, we use MovieLens dataset for only the consumer-side fairness for this experiment and the Kiva dataset for the provider-side fairness. We choose the movie genre as our protected attribute. One can use any other movie attribute. For example \cite{kamishima2018recommendation} chose the release year of the movie as the protected attribute.
        
        It can be seen in this data that there is a minority of female users (1709 out of the total of 6040). Certain genres display a discrepancy in recommendation delivery to male and female users. For example, in the ``Crime'' genre, female users rate a very similar number of movies (average of 0.048\% of female profiles vs 0.049\% of male profiles) and rate them similarly: an average rating of 3.7 for both female and male users. However, our baseline unmodified SLIM-U algorithm recommends in the top 10 an average of 1.10 ``Crime'' movies per female user as opposed to 1.18 such movies to male users. 
        
        Given that the rating profiles are similar but the recommendation outcomes are different, we can therefore conclude that the female users experience a deprivation of ``Crime'' movies compared to their male counter-parts. Similar losses can be observed for other genres. We are not asserting that there is any harm associated with this outcome. It is sufficient that these differences allow us to validate the properties of the BN-SLIM-U algorithm.
        
        In order to achieve genre discrepancy, within the MovieLens 1M dataset, we calculated the equity score (explained in \todo[inline]{where?}) and selected the five genres on which the baseline SLIM-U algorithm produced the lowest equity scores (e.i. $E_c@k$) : ``Film-Noir'', ``Mystery'', ``Horror'', ``Documentary'', and ``Crime''. Note that low equity scores are undesirable.
        
        \todo[inline]{should I summarize these explanations in the bln-slim chapter? if not, do I need to add a picture here?}



    \subsection{The Movies Dataset}
    
    This dataset is also a movie recommendation dataset which was obtained from the Kaggle website. It contains the metadata of 45,000 movies listed in the full MovieLens Dataset \footnote{https://grouplens.org/datasets/movielens} which were released on or before July 2017. As we mentioned before, movies are not a domain to which important fairness concerns are typically applied, we use this dataset as a well-known example with a rich set of provider-side features. The dataset contains 26 million ratings from 270,000 users for all 45,000 movies. Ratings are on a scale of 1-5.
    
    From the movies' set of features, the following were used in this project: genres, original language, release date, revenue, run-time, popularity, production countries and spoken language. A sample of this dataset was extracted which contained the 559,070 ratings from 6,000 users on 14,623 items (density of 0.63\%).

    All the selected features were then transformed into categorical variables. If the movie's popularity is greater than the average popularity, we tag the movie as popular and unpopular otherwise. We transform the revenue and run-time in the same way as well. The release date is bucketed into old and new if the movie's release date is before or after 1990 \cite{kamishima2016model}. All the categorical features were transformed into dummy variables, resulting in a total of 323 binary features (21 genres, 2 popularity buckets (popular and unpopular), 2 year buckets (old, and new), 2 revenue buckets (low, and high), 2 run-time buckets (short, and long), 133 languages and 161 production countries).
    
        \subsubsection{Protected feature(s)}

        This dataset was used in Chapter \ref{fairness_postproc} \todo[inline]{reference or add info about the dynamic fairness thing} to incorporate a multi-concern fairness definition in a post-processing approach. In other words, the goal is to mitigate unfair exposure (either too high for some and too low for some) in multiple sensitive attributes (e.g. country, and economic sector, and gender, and etc.) for the providers at the same time. This dataset was also used in \todo[inline]{dynamic fairness}, to address all the fairness concerns one at a time.
    
        Since we need to address the unfairness issue in all of these features, we need to select the protected group within each one of them.
        We identified the following protected classes within each category: ``unpopular'' (popularity), ``lower revenue'' (revenue) , ``longer'' (running time), ``before 1990'' (release date), ``Horror'', ``Music'', ``Mystery'', ``History'' (genres), and ``CA'', ``ES'', ``DE'', ``HK'' (countries). These feature categories were chosen because they represented a minority within each category.
    
        This dataset is rich in movie metadata that one can use for the study of fairness in recommendation. As an example, this dataset contains the metadata about the cast and the crew for each movie: such as the gender (non-binary) information of the director(s) and the writer(s), the number of people from each gender in the director team or the writer team, etc. We didnot use this information in our experiments.

    \subsection{Kiva dataset}
    \todo[inline]{changing our to my and we to I}
    This dataset is a proprietary dataset obtained from Kiva Microloans, a microlending site. And some of the collected data was extracted from the Kiva.org site using the site's API\footnote{http://build.kiva.org/}.
    % protected value: higher than average number of unfunded loans
    
    It includes all lending transactions over a 12-month period. Initially, there were 1,084,521 transactions involving 122,464 loans and 207,875 Kiva users.
    
    Of the 122,464 loans, we found that 116,650 were funded, that is they received their full funding amount from Kiva users by the 30-day deadline imposed by the site. Loans not fully funded within 30 days are dropped from the system and the money raised is returned to lenders.
    
    We selected only the funded loans for analysis. Each loan is specified by features including borrower's name/id, gender, borrower's country, loan purpose, funded date, posted date, loan amount, loan sector, and geographical coordinates such as geo-location, town, country and region of the borrower and the number of lenders who lent money to this loan.
    
    To reduce the feature space, and to solve the multicollinearity problem, highly correlated features were removed. The percentage funding rate (PFR) was added as a new feature, computed as follows:
    
    \begin{equation}
     \mbox{\textit{PFR}} =  \frac{1}{\mbox{\textit{{\#} days to fund}}} * 100 
    \end{equation}
    
    The percentage funding rate captures the speed at which a loan goes from being introduced in the system to being fully funded. For example, a loan with PFR of 25\% is accumulating a quarter of its needed capital each day. After preparing the data, the final features for each loan reduced to borrower's gender, borrower's country, loan purpose, loan amount (binned to 10 equal-sized buckets), and loan's percentage funding rate. 
    
    The transaction file included the lender's hashed id, lender's username, loan id, time of "purchase" or lending, the lent amount by that user, and the lending team id that that user is part of. A lender can be a part of one or several lending teams. These teams advocate for a specific cause for example agriculture, or are in support of a specific country such as Vietnam. We removed the lender user id, lending team id and the time of purchase id from the transactions file. The final columns were user's hashed id, loan id and the lent amount.
    
    
    We found that this dataset was highly sparse (density = $4.2e^{-5}$) and could not support effective collaborative recommendation, because a loan can only attract a limited amount of support (up to that needed for its funding). There are no ``blockbuster'' loans with thousands of lenders.
    
    To generate a denser dataset with greater potential for user profile overlap, we applied a content-based technique creating \textit{pseudo-items} that represent groups of items with shared features. We applied agglomerative hierarchical clustering \cite{rokach2005clustering} using the features of borrower gender, borrower country, loan purpose, loan amount (binned to 10 equal-sized buckets), and percentage funding rate (4 equal-sized buckets). We chose the cluster with the highest Silhouette Coefficient \cite{rousseeuw1987silhouettes} of around 0.69 which indicates a reasonable cohesion of the clusters. Note that there weren't any singleton clusters.
    
    After creating the feature file for psuedo loans, we transformed the lent amounts to ratings. We replace the loan id with the pseudo-loan id if that loan id exists in that cluster. Also, if a lender has lent to multiple loans in the same cluster(pseudo-loan), summation of all the lent amount is considered as one transaction since it shows how strong the interest of that lender to that item is.
    To transform the loan amount to ratings, for every user, we collected all the transactions on the pseudo-loans she has interacted with. We collected the lent amounts to each pseudo-loan and finally normalized it using the min-max normalization method in scikit-learn \cite{scikit-learn} to values between 3.0 and 5.0. Since ratings of 1s and 2s are considered as dislikes, since, if a lender hadn't liked a loan, she/he probably wouldn't have lent it any amount of money. Therefore, I set the minimum to 3.0. The min-max normalization is computed for every user separately over all her/his lent amounts.

    To make the dataset denser, we applied a 10-core transformation, selecting pseudo-items with at least 10 lenders who had funded at least 10 pseudo-items. The retained dataset has 2,673 pseudo-items, 4,005 lenders and 110,371 ratings / lending actions.
    
        % \subsubsection{kiva data collected from api}
        % The provider information in this Kiva dataset were collected from Kiva's API in September 2016. It contains approximately 1 million loans funded by approximately 180,000 lenders.

    \todo[inline]{true?}
    The final condensed dataset that were used in our experiments includes the pseudo items and the ratings behavior of lenders with them. Loan requests contain information about the borrowers, but the process of pseudo item creation hides and combines that information with the information of many other borrowers, therefore it protects the privacy of the borrowers. 
    
    Lender's information such as user id is hashed, and their rating behavior due to the aggregation of the loans, is different to some extent. For example, the transactions are precise, but the amount of the transaction could be different.
    As the privacy of the borrowers and the lenders are now protected to a great degree, with the permission of Kiva, I have made this dataset available to use for the research community. This version of the Kiva dataset, is one of my contributions in this thesis.


        \subsubsection{Protected features}
        In this dataset, we observed an imbalance within the following feature values/dimensions: (percentage funding rate), (country), (economic sector), (loan amount), (borrower gender). In keeping with Kiva's mission of providing equal access to capital across regions and economic sectors, we designate the items from the sectors and countries that have less than 1\% frequency in the training data as the protected group. As an example, 5 loan purposes in the economic sectors and 23 countries were selected to be the protected group. Note that we use Kiva dataset only for mitigating \textit{provider-side} unfairness either in one dimension or multiple-dimensions.
        
        We have used all these "fairness concerns" (or dimensions) in the OFAiR method explained in Chapter \ref{fairness_postproc} since OFAiR addresses multiple fairness concerns simultaneously. In Chapter \todo[inline]{where is dynamic fairness?}, we use all the previously mentioned fairness concerns, but unlike OFAiR, the proposed method, tries to mitigate bias in each fairness concern on at a time.
        
        In some of our research projects we have considered (region) instead of (country). Since these two features are highly correlated they both carry the same information and can be used interchangeably. In Chapter \ref{fairness_postproc} in the FAR/OFAR method and in Chapter \ref{fairness_inproc} in the Balanced neighborhood SLIM approach, we target the (region) feature.
        
        Kiva.org divides its borrowers into 9 geographic regions. We are defining the protected group as those regions of the world where it appears to be more difficult to fund loans. (In Kiva.org, a loan that does not attract enough lenders over a 30 day period is marked as unfunded and dropped from the system.)\todo{redundant?} Therefore, we are promoting the loans that have not grabbed the lenders attention so far and help them get funded faster. Note that the assumption behind this method is that all the loan requests are equally worthy regardless of all their information, in other words the more popular loans are not inherently better than the less popular ones.
        
        In our dataset, we find that there are some geographic regions with a higher than average number of unfunded loans. In these regions, borrowers have a lower probability of getting the desired capital. As shown in Table~\ref{tab:unfunded}, the regions of North America, Eastern Europe, South America, and Asia have proportionately more funded loans than the regions of Africa, Middle East, and Central America\footnote{Our data set had only a single loan request from Australia.}. These regions where borrowers have lower funding percentages are treated as the protected group in our experiments.
        
        \begin{table}
            \centering
        \begin{tabular}{l|l|r}
            Category & Region & Unfunded \% \\ \hline
            Unprotected & North America & 1.73 \\
            & Eastern Europe & 0.99 \\
            & South America & 4.33 \\
            & Asia & 6.70 \\ \hline
            Protected & Africa & 10.57 \\
            & Middle East & 13.23 \\
            & Central America & 8.81 \\
        \end{tabular}
            \caption{Percentage of unfunded loans by region}
            \label{tab:unfunded}
        \end{table}
 


    \subsection{Yelp dataset}
    \todo[inline]{in the Hypertext paper we used this data for mis-calibaration analysis}
    The original Yelp.com data included 6.7 million reviews from 1.6 million users on 192,609 businesses from different industries. We used a filtered subset, Yelp\_core40 \cite{mansoury2019bias}, which includes users who rated at least 40 businesses and businesses which were rated by at least 40 users. In the Yelp\_core40, there are many business categories  (e.g. car dealers, airports, or different types of restaurants, etc.). We limited the data set to restaurant/food establishments, using both "restaurant" and "food" labels for filtering. After pre-processing, the data set included 85,041 ratings by 1,355 users over 1,077 restaurants, covering 231 categories. The sparsity of Yelp is 5.83\%. The categories are defined by the labels provided in the original Yelp data set. Some of the categories are very popular (e.g. "American (New)", "Breakfast \& Brunch", etc.), while some of them are extremely specific (e.g. "Pretzels", "Shaved snow", etc.). There were also many establishments with multiple labels, which meant that they belonged to different item categories.\todo{protected features?}



\section{Evaluation Metrics}
\label{sec:eval}
    
    Recommender systems can be evaluated based on user-studies, online or offline methods.
    
    In user studies, test subjects are recruited and are asked for their feedbacks about the system and about their interaction with it. Then this data is used to infer the likes and dislikes of users. 
    In an online system, we measure user reactions with respect to the presented recommendation lists such as conversion rate of users. Such testing methods are referred to as A/B testing and they measure the direct impact of recommendations on the end users. In this type of testing active user participation is essential although it is not always feasible to have access to this data from online systems in various domains, especially in academia. In such cases, offline methods with historical data, such as ratings, are used. Since we lack access to real-time user interaction data, offline evaluation in this dissertation which is the most common approach to evaluate recommender systems in this situation. We have also made use of the user-study based evaluation to investigate \todo[inline]{user-study fairness} how users perceive fairness in recommender systems.
    
    In order to obtain an understanding of the effectiveness of recommender systems, we need to make use of proper evaluation metrics. Multiple metrics are needed because the evaluation of recommender systems is often multifaceted, and a single criterion cannot capture all of the goals of a designer. As an example, accuracy-based metrics show the quality of the recommendations that users receive on average, but it can't reflect whether we have a diverse set of providers in the recommendations or not. 
    
    Our goal in this dissertation is to design methods that increase the tradeoff between accuracy and fairness, therefore we need to use metrics that assess both the accuracy and the unfairness of the output. We describe these two categories below.
    
    Since all of the proposed metrics were either ranking or re-ranking approaches, we used the following ranking metrics to measure the accuracy of our approaches: nDCG, Precision, and Recall.
    
    To estimate the unfairness of the output, we used the following fairness metrics and fairness inducing diversity metrics: Statistical Parity@k (Equal Chance), Intra-list distance(ILD), Average Coverage Rate (ACR) for Provider-side fairness, provider-side Risk Ratio, protected item exposure, equality of odds/chance, and Miscalibration error.
    \todo[inline]{check the list again}
    
    
    \subsection{Accuracy metrics}
    
    Accuracy metrics are the single most important component in the evaluation of recommender systems. Although, secondary metrics such as fairness play a key role in understanding the social aspects of a recommendation system, accuracy metrics are still prioritized. Accuracy measures the ability of a recommender system to either predict the ratings or the correctness of ranking the items in users' recommendation lists. We evaluate the ranking accuracy of our algorithms using the following common metrics below. Assume that one selects the top-$k$ set of ranked items to recommend to the user.
    
        \textbf{Precision}: For any given value $k$ of the size of the recommended list, the set of recommended items is denoted by $\mathcal{S}(t)$. Note that $\mathcal{S}(t)=k$. So, as $k$ changes, the size of $\mathcal{S}(t)$ changes too.
        
        Let $\mathcal{G}$ represent the true set of relevant items (ground-truth) that are consumed by the user. Then, for any given size $k$ of the recommended list, the precision is defined as the proportion or percentage of recommended items that are truly relevant (i.e., clicked, viewed or consumed by the user). 
        
        \begin{equation}
        \text{Precision(k)}=\frac{|\mathcal{S}(k) \displaystyle \cap \mathcal{G}|}{|\mathcal{S}(k)|}
        \label{eq:precision}
        \end{equation}
        
        
        \textbf{Recall}
        The recall is defined as the proportion of the set of relevant items (ground-truth) that have been recommended for a list of size $k$.
        
        \begin{equation}
        \text{Recall(k)}=\frac{|\mathcal{S}(k) \displaystyle \cap \mathcal{G}|}{\mathcal{G}}
        \label{eq:recall}
        \end{equation}
        
        Similarly to precision, these values can be described as percentages.
        
        \todo[inline]{delete this totally?}
        \textbf{Normalized Discounted Cumulative Gain (nDCG)}: Proposed in~\cite{jarvelin2002cumulated}, nDCG is a commonly-used measure of ranking quality, defined as Eq.~\eqref{eq:ndcg}. In this measure, an item appearing on a recommendation list accrues ``gain'' according to its position on the list -- thus the discount. The measure is normalized by comparing the algorithm's performance to the best ranking that could have been achieved. 

        \begin{equation}
        \text{nDCG}=\frac{\text{DCG}}{\text{IDCG}},
        \label{eq:ndcg}
        \end{equation}
        
        where the formula of Discounted Cumulative Gain (DCG) of a $K$ ranked list is defined as
        
        \begin{equation}
        \text{DCG}=\sum_{i=1}^K\frac{{rel}_i}{\log_2(i+1)},
        \end{equation}
        
        where ${rel}_i$ is defined by the ratings of the item in the test set at position $i$. Ideal Discounted Cumulative Gain (IDCG) is the maximum possible DCG, which is the value of DCG computed by sorting all the items in the test set by their ratings.
        We measure the normalized discounted cumulative gain at $k=10$ (list length). In some of the research projects ${rel}_i$ is considered as an indicator function that is 1 for items that the user liked and 0 for others.
        
    
    \subsection{Fairness metrics}
    
        % it's essential that we consider the target stakeholders, the definition of harm or unfairness and the specific metrics for measuring harm or integrating in the system to avoid harm.
        
        \todo[inline]{robin says concepts will be in chapter 2?}
        \todo[inline]{needs an intro}
        I have introduced some of the following metrics.: ACR, DPF
        mostly provider-side fairness.
        all group-based metrics.
        
        exposure based metrics: ACR
        accuracy-based metrics: DPF (DPPF, DPCF)
        
        
        
        \textbf{Average Coverage Rate (ACR)}: To measure the provider-side fairness, we compute the average number provider groups covered by the ranked list. This metrics was initially designed to be used in microlending application.
        % in microlending application, we compute the average number of borrower groups covered by the ranked list.
        
        \begin{equation}
            \text{ACR}=\frac{\sum_{u\in U_t}N_{S(u)}}{N_\text{pg}|U_t|},
        \end{equation}
        
        where $U_{test}$ is the test user set, $|U_{test}|$ is the number of users in the test set, $N_\text{pg}$ is the total number of provider groups ($pg$) and $N_{\mathcal{S}(u)}$ is the number of provider groups covered in the list $\mathcal{S}(u)$. A larger ACR indicates a better coverage of the provider groups thus a higher provider-side fairness in the system.
        
        
        \todo[inline]{re-visit with care. it doesn't seem to be correct at the moment!}
        \textbf{Weighted Proportional Fairness ($WPF$)}: Here we introduce a well-accepted and axiomatically justified metric of fairness, the weighted proportional fairness \cite{kelly1998rate}. Weighted proportional fairness is a generalized Nash solution for multiple groups.
        
        \begin{definition}[Weighted Proportional Fairness]
        An allocation of desired activities $x_t$ is weighted proportionally fair if it is the solution of the following optimization problem,
        \begin{equation}\label{eq:fairness}
            \max_{x_t}\,\, \sum_{i=1}^lw_i\log(x^i_t), \quad\text{s.t.}\,\,\sum_{i=1}^lx^i_t = 1,\, x^i_t\geq0,\,i=1,\ldots,l.
        \end{equation}
        \end{definition}
        
        % Where $x_c$ denotes a provider group with the sensitive attribute $c$.
        The coefficient $w_i\in\mathbb R_+$ is a pre-defined parameter weighing the importance of each group.
        
        
        \todo[inline]{is the formula correct recheck? change the notation i. confused!}
        \begin{equation*}
        x_{t}^i=\sum_{u\in\mathcal U}\sum_{t}\sum_{i=1}^K\frac{{rel}_i\mathbbm{1}_{\{t\in \mathcal X_{t}\}}}{\log(i+1)}
        \end{equation*}
        
        
        The optimal solution can be easily solved by standard Lagrangian multiplier methods, namely
        
        \begin{equation}
        x_*^i=\frac{w_i}{\sum_{i^{'}=1}^lw_{i^{'}}}.
        \end{equation}
        
        Higher values of $WPF$ are better. The optimal value is obtained when the allocation is proportional to the assigned weights. For example, let assume the weights are $w = [w_{c1}, w_{c2}] = [0.8, 0.2]$. If the utility (allocation) is uniformly distributed $x = [x_{c1}, x_{c2}] = [0.5,0.5]$, the metric is computed as $dpf = 0.8*\log 0.5+0.2*\log 0.5=-0.301$. The maximum is achieved when $x = [x1, x2] = [0.8, 0.2]$, where $dpf=0.8\times \log 0.8 + 0.2\times \log 0.2 = -0.217 $. In other words, the maximum value is achieved when each group is assigned a utility that is equal to its weight.
        
        The weights can be a hyper-parameter and can be defined according to the popularity of each group, the system preferences on who deserves to have a higher weight, or any other method.
        
        % dividing it by the max utility or ideal utility when the weights and the utilities are the same.
        We have implemented this metric in \libauto{} (explained in Chapter \ref{librec-auto}) for both providers and the consumers.
        
        
        
        
        \textbf{Normalized Discounted Proportional Fairness ($NDPF$)}

        \todo[inline]{look at Weiwen's latest paper that uses a version of this measure}
        % Here we introduce the discounted proportional fairness metric which we adopted based on the well-accepted and axiomatically justified metric of fairness, the proportional fairness \cite{kelly1998rate}. Proportional fairness is a generalized Nash solution for multiple groups.
        Ideal Discounted proportional fairness is achieved when
        \begin{equation}
        x_*^i=\frac{w_i}{\sum_{i^{'}=1}^lw_{i^{'}}}.
        \end{equation}
        
        therefore,
        
        \begin{equation*}
            \text{$IDPF$}=\sum_{c=1}^{n_c} w_c \log\left(x_*^i\right),
        \end{equation*}
        
        
        \begin{equation*}
            \text{$DPF$}=\sum_{c=1}^{n_c} w_c \log\left(\frac{x_c}{\sum_{c'}x_{c'}}\right),
        \end{equation*}
        
        \todo[inline]{recheck the notations}
        where $c$ is representing a provider group $\mathcal V_c$ with a specific sensitive attribute, $n_c$ is the total number of providers, $x_c$ is the utility (allocation) of a provider group. We define the utility of $\mathcal V_c$ by the cumulative gain that $\mathcal V_c$ received from all users,
        
        \todo[inline]{is the formula correct recheck?}
        \begin{equation*}
        x_c=\sum_{u\in\mathcal U}\sum_{c}\sum_{i=1}^K\frac{{rel}_i\mathbbm{1}_{\{v\in \mathcal V_{c}\}}}{\log(i+1)}
        \end{equation*}
        
        \begin{equation*}
            NDPF = \frac{DPF}{IDPF}
        \end{equation*}

        This metric is bounded between 0 and 1. Therefore easier to interpret and understand.
        
        
        
        \textbf{Consumer-side Risk Ratio}: In evaluating fairness of outcome, we use a variant of what is known in statistics as \textit{risk ratio} or \textit{relative risk} (RR)\cite{romei2014multidisciplinary}. We measure what is effectively \textit{relative opportunity}. In other words, we measure the observed probability of protected class items being recommended divided by the probability of unprotected class items being recommended.
        
        We construct a consumer-side equity score, $E_c@k$ for recommendation lists of $k$ items, as the ratio between the outcomes for the different groups. Let $P_i@k = {\rho_1, \rho_2, ..., \rho_k}$ be the top $k$ recommendation list for user $i$, and let $\gamma()$ be a function $\rho \rightarrow \{0,1\}$ that maps to 1 if the recommended item is in a protected group. Then:

        \begin{equation}
        E_c@k=\frac{\sum_{i \in U^+}{\sum_{\rho \in P_i@k}{\gamma(\rho)}}/|U^+|}
        {\sum_{i \in U^-}{\sum_{\rho \in P_i@k}{\gamma(\rho)}}/|U^-|}
        \end{equation}
        
        
        \todo[inline]{should I make the metrics general? or specific as in our cases? for example use item instead of movies?}
        $E_c@k$ will be less than 1 when the protected group is, on average, recommended fewer items that they desired. It may be unrealistic to imagine that this value should approach 1: the metric does not correct for other factors that might influence this score -- in the case of movie recommendation for example, female users may rate a particular genre significantly lower and an equality of outcome should not be expected.
        While the absolute value of the metric may be difficult to interpret, it is still useful for comparing algorithms. Higher $E_c@k$ than 1 means that the items which are favorable by the protected group are recommended more than necessary, to the protected group. In other words, we might have reversed the unfairness. Note that this is an additive, utilitarian measure of outcome equity and does not take into account variations in user experience. More nuanced measures of distributional equity, including Pareto improvement, we leave for future work.
        
        
        \textbf{Provider-side Risk Ratio}: The provider-side equity score, $E_p@k$, is defined on recommendation lists of $k$ items. Let $L^+$ be the set of items in the test set that are from the protected groups, and $L^-$ be the corresponding set from the unprotected groups. Also, let $\pi^+()$ be an indicator function $\rho \rightarrow \{0,1\}$ that maps to 1 if the recommended item is from a protected group and $\pi^-$ is a similar function for the unprotected groups. Then:

        \begin{equation}
        E_p@k=\frac{\sum_{i \in U}{\sum_{\rho \in P_i@k}{\pi^+(\rho)}}/|L^+|}
        {\sum_{i \in U}{\sum_{\rho \in P_i@k}{\pi^-(\rho)}}/|L^-|}
        \end{equation}
        
        $E_p@k$ will be less than 1 when items from the protected group are appearing less often on recommendation lists. As with $E_c$, this is a utilitarian measure, summing over all providers, and does not speak to the the distribution across individual providers. Like $E_c$, it does not take the rank of recommended items into account.


        \textbf{Protected group hit-rate}
        The fairness of lists were evaluated based on protected group hit rate, which measures the ratio of the protected group to appear in a recommendation list. This value is related to the fairness concept of ``statistical parity,'' measured relative to items' level of promotion within the recommender system. Because list lengths are fixed ($k$ is set in our experiements), the hit rate for unprotected items is just one minus the protected group hit rate. 
        \todo[inline]{add the following section to the paper itself}
        % We will calculate this metric for every 1\% accuracy loss to have a better understanding about the accuracy / fairness trade off.
        
    
        \textbf{Misalibration error}
        \todo[inline]{do we need this part? drop?}
        
        The concept of miscalibration in recommendations is introduced in \cite{steck2018calibrated} by Steck. This metric measures whether recommender systems reflect the various interests of users relative to their initial preference proportions or not. The greater the difference between a user initial preferences and her recommendations, the higher the error. In other words the users with high miscalibration error, are not receiving the recommendations that match what they want.
        
        Miscalibration error often disproportionately impacts users, therefore has a close relationship with fairness. In our projects we have used the miscalibration error as a consumer-side fairness metric.
        
        
        Steck uses Kullback-Leibler (KL) divergence to measure the difference between the probability distributions (preference distributions) across item categories, in a user profile and the distribution in the user's recommendation set. Both, are based on the distribution of item categories $c$ for each item $i$, denoted by $p(c|i)$.
    
        \begin{itemize}
            \item $p(c|u)$: the distribution over categories $c$ of the set of items $\mathcal{H}_u$ interacted with by user $u$ in the past.
            \begin{equation}\label{input_preference}
                p(c|u) = \frac{\sum_{i \in \mathcal{H}_u} w_{u, i} \cdot p(c|i)}{\sum_{i \in \mathcal{H}_u} w_{u, i}},
            \end{equation}
        
            where $w_{i,u}$ is the weight of each item $i$, e.g. how recently it was liked or clicked on, or its popularity or rank.
            
            \item $q(c|u)$: the distribution across categories $c$ of the list of items recommended to user $u$.
            \begin{equation}
                q(c|u) = \frac{\sum_{i \in \mathcal{I}_u} w_{r(i)} \cdot p(c|i)}{\sum_{i \in \mathcal{I}_u} w_{r(i)}},
            \end{equation}
            
            where $\mathcal{I}_u$ is the set of recommended items and $w_{r(i)}$ is the weight of an item and can be measured by its rank $r(i)$ in the recommendation list.
        \end{itemize}
    
        KL-divergence \cite{kullback1997information} is used to measure the difference between these two probability distributions, or the divergence of $p$ from $q$. KL-divergence is denoted by:
        \begin{equation} \label{kl}
        MC_{KL}(p||q) = KL(p||\Tilde{q})= \sum_{c \in C}{p(c|u)\log\frac{p(c|u)}{\Tilde{q}(c|u)}},
        \end{equation}
        
        where $p(c|u)$ is the target distribution. If $q$ is similar to $p$, $MC_{KL}$ will take small values, and in the case of perfect calibration, it is 0. $MC_{KL}$ diverges if a category $c$ is $q(c|u)=0$ and $p(c|u)>0$, so instead we use:
        \begin{equation}
            \Tilde{q}(c|u) = (1 - \alpha) \cdot q(c|u) + \alpha \cdot p(c|u),
        \end{equation}
        
        where $0 < \alpha < 1$, so that $q \approx \Tilde{q}$. We set $\alpha = 0.01$ in this experiment.
        
        I renamed this metric to $MC_{KL}$ instead of $C_{KL}$ which is described in \cite{steck2018calibrated}, since it specifies the degree to which we have miscalibration in our recommendations and it is more in line with the values that KL-divergence takes. For example, if $p$ and $q$ are very similar, KL-divergence takes lower values, so miscalibration is low and vice versa.
        
        One properties discussed in \cite{steck2018calibrated} is worth mentioning. $MC_{KL}$ is sensitive to small differences when $p$ is small. For example, if a user liked a category 2\% of the time and it is recommended to her 1\% of the time, $MC_{KL}$ considers it a significant change compared to a situation where a user likes a category 50\% of the time, while it's recommended to her 49\% of the time. 
    
        \todo[inline]{we need a summary of evaluation metrics in a table or not?}


        \textbf{Intra-list distance(ILD)}: We used intra-list similarity as introduced by Ziegler et al. \cite{ziegler2005improving} to measure the diversity of a list. Diversity here is the measure of dissimilarity between items in a set. This measure uses average pairwise distance of items in a set. We adjusted this metric to compute the diversity based on the item features in a list of each user rather than calculating the diversity of items based on their ratings. This list could be the recommendation list or the user profile.
        
        Therefore, ILD is a feature-based diversity metric that is used as a provider-side and exposure-based fairness metric.
        
        \todo[inline]{is it a catalogue coverage metric?}
        
        \begin{equation}
            \text{ILD}(\mathcal{L}) = \frac{1} {|\mathcal{L}|(|\mathcal{L}|-1)} \sum_{i \in \mathcal{L}}\sum_{j \in \mathcal{L}}d(i,j)
        \end{equation}
        Higher values in this metric means a more comprehensive coverage of the items that are different in their features, yet still desirable.
        \todo[inline]{why is this a fairness metric?}



%%%%%%%%%%%%%%%%%%%%%%%%%%%%%%%%%%%%%%%%%%%%%%%%

        \textbf{Shannon Entropy}:
        \todo[inline]{robin says what is going on here?}
        
        The concept of information entropy was introduced by Claude Shannon in his 1948 paper \cite{entropy1948} and is also referred to as Shannon entropy. In information theory, the entropy of a random variable is the average level of "information", "uncertainty" or, "impurity" inherent in the variable's possible outcomes. The entropy measures the “amount of information” present in a variable. That is, the more certain or the more deterministic an event is, the less information it will contain. In a nutshell, the information is an increase in uncertainty or entropy. 
        
        Given a discrete random variable $X$, with possible outcomes $x_{1},...,x_{n}$, which occur with probability $P(x_1),...,P(x_n)$
         the entropy of $X$ is formally defined as:
         
         \begin{equation}
            H(X)=-\sum_{i=1}^{n}P(x_{i})\log P(x_{i})
         \end{equation}

        where $\Sigma$ denotes the sum over the variable's possible values ($n$). The value of entropy always lies in the range (0, 1).
        
        We use this concept in two ways in our work: (a) to measure the diversity in a recommendation list and (b) to learn more about user preferences.
        
        Entropy is a measure of impurity or heterogeneity. In the recommender system literature it is known as a metric to measure the diversity in a list or a.k.a distributional inequality \todo{cite}. It can be computed over different aspects of a list: ratings, item features, or etc.
        Higher entropy, means higher heterogeneity or diversity in that list and lower entropy shows more homogeneity in a list. As an example, it can measure the uncertainty of an item’s rating (output value) in data. For example, it captures the variability of ratings over a certain movie by all the users in the dataset. Or it can compute the diversity of the movie genres in a user's recommendation list. We have used this metric in Chapter \ref{fairness_postproc} \todo{refer to the chapter you used it in}
        
        We have also used entropy to learn more about the users and to increase personalization. By calculating the entropy of the items that a user has interacted with (user's profile) we can learn the level of diversity she prefers in her list. We can also measure the variability of recommendations made to a user as the user profile changes. 
        As an example, if the items in a user's profile are more alike (e.g. same genres of movies like Romance), the entropy will be lower. In other words the user's preference for specific movie genres is more intense or more deterministic, hence she might not like the system to diversify her recommendation list.
        
        On the other hand, when a user interacts with various dissimilar items (e.g. different movie genres such as Romance, Historical, Children, etc.), the entropy is higher which indicates that the preferences of the user over different categories are more equally distributed. Therefore the entropy
        
        will probably like her recommendations to be diverse as well. We use this concept in Chapter \ref{fairness_postproc} to increase provider-fairness, which will be explained in details.
        
        % This concept is used in user modeling to increase personalization. 
        
        % \begin{equation}
        %     \text{Entropy}(i) = - \sum_{i=1}^n p(i) \log p(i)
        % \end{equation}
        % let $n$ be the total number of items, with smaller values being more indicative of discriminative power.
        
        % Entropy is also a measure of catalog coverage which is used to compute the diversity of a list. The term coverage refers to the proportion of items that the recommendation system can recommend. 
    
        % \begin{equation} \label{upe_c}
        % E(u) = -\sum_{c \in C}{p(c|u)\log{p(c|u)}}
        % \end{equation}
    

        
        % \begin{equation} \label{upe_i}
        % E(u) = -\sum_{i \in  \mathcal{H}_u}{p(i|u)\log{p(i|u)}}
        % \end{equation}
        
        % where: $p(i|u) = \frac{w(u,i)}{\sum_{i \in \mathcal{H}_u}{w(u,i)}}$ with $\mathcal{H}_u$ here and in the following factor presenting the rated items in the user's profile. 
            


\section{Experimental approach}
    % n-fold, data cleaning, pre-processing 7 filtering, train/test etc.
    % all the hyper parameters, librec-auto
    
    \subsection{initial Setup}
    
    % All the datasets were split into 80\% for training data and the remaining 20\% for the tested.
    % In the training phase of all the algorithms \todo[inline]{all?most}, 5-fold cross validation was used.
    All experiments were performed using a 5-fold cross validation setting where 80\% of each user's rating data is used for the training data set and the rest is used as the test data set (LibRec's \texttt{userfixed} configuration). The training set was used for building a recommendation model and generating recommendation lists, and the test set was used for evaluating
    the performance of generated recommendations. A random seed was set for all the experiments to ensure the repeatability of the algorithms.
    
    For evaluation purposes of our ranking methods, we chose $k = 10$ (top 10 recommended items). Each recommendation algorithm can have several hyper-parameters. To identify the best-performing sets of hyper-parameters for each algorithms, I performed grid-search on hyper-parameters space and selected the results with the highest precision for the next analysis. 
    
    
    and for each algorithm we chose the hyper-parameters that achieved the highest performance using the Grid Search method.
    
    For the post-processing approaches that we propose, the baseline recommender generates a long recommendation list for users (e.i. 200), then the post-processing approaches selects the top@k for each user according to their selection strategy.
    
    I used librec-auto and LibRec for running the experiments\cite{burke2020facct_libauto,Sonboli2020FARLA,guo2015librec,mansoury2019algorithm,mansoury2018automating}. I extended \libauto{} and LibRec to include all the fairness-aware metrics, used diversity metrics that I have used or proposed as well as the re-ranking algorithms and recommendation algorithms and integrated. Evaluation metrics are thoroughly explained in Section \ref{sec:eval}.

    
    \subsection{baseline recommendations}
    
        The following baseline recommendation systems were used to indicate the improvement of the proposed approaches in comparison with these well-established algorithms: most-popular, item-based $K$ nearest neighbor, non-negative matrix factorization. 
        
        In Chapter \ref{fairness_postproc}\todo[inline]{mention the section as well}, we explain about the re-ranking algorithms that we considered as baselines. Recommendation system baselines that are close to the proposed approaches are explained in each chapter.
        
        
        \textbf{Most popular item}
        
        Most popular baseline recommends the most popular items in the dataset in order of their popularity. This is a non-personalized recommender, and falls short in generating good recommendations. This popularity-based model is often used in the cold-start setting when we don't have any ratings from the users. Although, this baseline is not always easy to beat due to the popularity bias that usually exists in the recommendation datasets. In other words, in datasets with high popularity bias, a significant part of the ratings can be explained by item popularity, rather than any specific personalized preferences of users for items. So, it is important to have this non-personalized algorithm as a baseline to ensure the outcomes of the proposed algorithm is not only influenced by the popularity bias in the input data.
        
        
        \textbf{Item-based K-nearest-neighbor (item-Knn)}: 
        
        Nearest neighbor models are a type of collaborative filtering models that use the collaborative power of the ratings provided by multiple users to make recommendations. In this approach the ratings of users on items are predicted on the basis of their neighborhoods. Neighborhoods can be the user nearest neighbors, who are similar to the target user or they can be the nearest neighbors of a target item which are the items most similar to that target item.
        
        In item-Knn order to make the rating predictions for target item B by user A, the first step is to determine a set S of items that are most similar to target item B. The ratings in item set S, which are specified by A, are used to predict whether the user A will like item B. Therefore, a person's ratings on similar science fiction movies like Alien and Predator can be used to predict her rating on Terminator.
    
        % For the $m\timesn$ ratings matrix $R = [r_{uj}]$ with $m$ users and $n$ items, let $I_u$ denote the set of item indices for which ratings have been specified by user (row) $u$ (user profile). 
        
        
        Consider the case in which the rating of target item $t$ for user $u$ needs to be determined. The first step is to determine the top-k most similar items to item $t$ based on the aforementioned adjusted cosine similarity. Let the top-k matching items to item $t$, for which the user $u$ has specified ratings, be denoted by $Q_{t}(u)$.

        The weighted average value of these ratings is reported as the predicted value $\hat{r}_{ut}$. The weight of item $j$ in this average is equal to the similarity between item $j$ and the target item $t$. 
        
        \begin{equation}
            \hat{r}_{ut} = \frac{\sum_{j \in Q_{t}(u)} \text{Similarity(j,t)} \cdot r_{uj}}{\sum_{j \in Q_{t}(u)} |\text{Similarity(j,t)}|}
        \end{equation}
        
        The underlying idea here is to make use of the user’s \textit{own} ratings on similar items in the final step of making the prediction. For example, in a movie recommendation system, the nearest neighbors of the movie, will typically be movies of a similar genre. The ratings history of the same user on such movies is a very reliable predictor of the interests of that user.
        
        We use a standard item-based nearest neighbor techniques \cite{desrosiers2011comprehensive} as a baseline recommendation system in our research. Neighborhood-based collaborative filtering algorithms were among the earliest and very effective collaborative filtering approaches and among the most popular methods. Their popularity might not be because of their accuracy but rather because they are quite simple to implement. It is also quite easy to explain their generated recommendations with the list of the neighbors used. Another strengths of neighborhood based models is its potential to make serendipitous recommendations that can lead users to the discovery of unexpected, yet very interesting items.
        
        When the data is sparse, these algorithms don't work very well as there aren't sufficiently similar neighbors to predict the rating robustly. Therefore they don't have a full coverage of the rating predictions. Of course, if we only focus on the top-K items, this problem seems to become less significant.
        
        Most of our proposed methods aim to increase provider-side fairness using the methods that increase diversity in the recommendation lists so we can then filter or re-order the items based on the system's fairness concerns. 
        
        Item-Knn tends to have a more diverse recommendation outcome compared to user-Knn, therefore it is a good baseline to have to compare the proposed methods against.
        
        User-Knn has a less diverse outcome mainly due to the popularity bias issue. In a movie recommendation scenario, this means that most of the users have rated popular genres of movies. Now, if a user likes a niche genre, since we might not have enough ratings of nearest neighbor users for that niche genre, the user-Knn algorithm might perform poorly in predicting the rating of that movie for that user. In a way user-Knn has a great prediction on the movies that we have a lot of data about, while ignoring the niche genres. This issue leads to have less diverse recommendation results. While item-Knn, predicts the rating for that niche genre based on the user's own ratings. SInce the user likes that genre, she might have rated more movies in that genre which gives the algorithm enough data to have a good prediction for the user. Int this way item-Knn keeps all the less popular genres in the recommendations, hence more diverse recommendation results. Therefore item-Knn is a better option as our baseline.
        
        % So, the data has popularity bias, therefore most of the users have rated popular items. So, If I like a niche genre that only a few people like, user-Knn might not be able to do a good job on predicting a rating for me about that niche genre. And it's only giving me great predictions on the items that a lot of people have rated.But since item-Knn calculates the ratings based on my own ratings, It should have a better prediction whether i like that movie or not.  Since I like that niche genre, I have probably rated some movies in that genre, and because it uses my own ratings, it's predictions are more accurate.So item-Knn has diversity since it includes all the preferences of users, but user-Knn ignores the less popular preferences of users therefore less diversity.
        
        Besides the previously mentioned reason, we chose item-based Knn, as they are typically preferred when the number of users far exceeds the number of available items since they provide more accurate recommendations in this case. Compared to item-based Knn, they are more computationally efficient and requiring less frequent updates.

       
        \textbf{Non-negative matrix factorization (NNMF)}:
       
       Matrix factorization models \cite{koren20009mf} are also in the collaborative filtering family of methods. In a basic matrix factorization model, the rating matrix $R$ of size $m\times n$ is approximately factorized into $m\times k$ matrix $U$ and and $n\times k$ matrix $V$, as below:
       \begin{equation}
           R\approx UV^T
       \end{equation}
       
       Each column of $U$ (or $V$ ) is referred to as a latent vector or latent component, whereas each row of $U$ (or $V$ ) is referred to as a latent factor. The $i$th row $\bar{u_i}$ of $U$ is referred to as a user factor, and each row $\bar{v_j}$ of V is referred to as an item factor. They both have $k$ entries and they correspond to the affinity of user $i$ and item $j$ to the $k$ concepts.
       Each rating $r_{ij}$ in in $R$ can be approximately expressed as a dot product of $i$th user factor and $j$th item factor:
       \begin{equation}
           r_{ij} \approx \sum_{s=1}^{k} u_{is} \cdot v_{js}
       \end{equation}
       
       Where $u_{is}$ is the affinity of user $i$ to concept $s$ and $v_{js}$ is the affinity of item $j$ to concept $s$. 
       Non-negative matrix factorization (NNMF) \cite{lee2001algorithms,zhang2006learning} may be used for rating matrices that are non-negative ($U$ and $V$). The major advantage of this method is its interpretability which helps in understanding the user-item interactions. This property is most useful for cases when the the data consists of implicit feedback data (a unary ratings matrix which only contains positive feedback) such as clicks or likes.
       
       We chose NNMF as one of our other baseline recommender systems as a well-established collaborative filtering algorithms as a the latent factor model and because of its higher interpretability compared to other matrix factorization models.
       
       
    %   non-negative matrix factorization model treats missing entries as negative feedback by setting them to 0s. This example is a manifestation of overfitting caused by lack of

        
        % NMF(non-negative matrix factorization), UserKNN, itemKNN, BPR WRMF, Maxent, RankSGD, SLIM, FAR/PFAR, most-popualr, MMR, variants of ofair
        
        % (a) RankSGD \cite{pmlr-v18-jahrer12b} uses stochastic gradient descent to optimize the ranking error; (b) UserKNN \cite{resnick1997recommender}  is a memory-based collaborative algorithm that computes user similarity; (c) Weighted Regularized Matrix Factorization (WRMF) ~\cite{hu2008collaborative} creates a reduced-dimensionality factorization of the rating matrix; (d) Maximum-entropy distribution (Maxent) \cite{choo2014gather} is a loan recommender system specially designed for Kiva. Maxent models lending behaviors by estimating a maximum-entropy distribution based on a set of heterogeneous information regarding micro-financial transactions available at Kiva.
        
        % We also used MMR by itself, as a diversity-enhancing re-ranker, a variant of OFAiR that includes only user tolerance weights for each feature, and a variant that includes only the fairness weights for the protected feature dimensions without the tolerance weights. In this way, we can study separately the contribution of each of these aspects of the algorithm.
        
        % The recommendation algorithms for generating the recommendation lists of size 10 in pre-processing solution are Biased Matrix Factorization (BiasedMF) [94], Singular Value Decomposition (SVD++) [93], and List-wise Matrix Factorization (ListRankMF) [160]. The recommendation algorithms for generating the longer recommendation lists of size 50 in post-processing solution are Bayesian Personalized Ranking (BPR) [146], Neural Collaborative Filtering (NCF) [71], User-based Collaborative Filtering (UserKNN) [147]. I chose these algorithms to cover different approaches in recommender systems: matrix factorization, neural networks, and neighborhood models.
    
        % user study to see how users understand fairness in recsys or transparency
        

    \subsection{notations}
        % Let U and I be the sets of users and items, respectively. The lists of recommendations is denoted as R. Ru is the recommendation items for user u ∈ U and user profile Iu is the list of items that u has rated

        \todo[inline]{needs to be revisited}
       
    %%%%%%%%%%%%%%%%%%%%%%%%%%%%%%%%%%%%%%%%%%%%%%%%%%%%%%%%%%%%%%%%%%%%%%%%%%%%%%%%%%%%%%%%%%%%%%%%%%%%%%%%%%%%%%%%%%%%%%%%%%%%%%%%%%%%%%%%%%%%
    % capital X for items and Y for users

    %%%%%%%%%%%%%%%%%%%%%%%%%%%%%%%%%%%%%%%%%%%%%%%%%%%%%%%%%%%%%%%%%%%%%%%%%%%%%%%%%%%%%%%%%%%%%%%%%%%%%%%%%%%%%%%%%%%%%%%%%%%%%%%%%%%%%%%%%%%%
    % Different recommendation scenarios can be distinguished by differing configurations of interests among the stakeholders. We divide the stakeholders of a given recommender system into three categories: consumers $C$, providers $P$, and platform or system $S$. The consumers are those who receive the recommendations. They are the individuals whose choice or search problems bring them to the platform, and who expect recommendations to satisfy those needs. The providers are those entities that supply or otherwise stand behind the recommended objects, and gain from the consumer's choice.\footnote{In some recommendation scenarios, like on-line dating, the consumers and providers are same individuals.} The final category is the platform itself, which has created the recommender system in order to match consumers with providers and has some means of gaining benefit from successfully doing so. 

    %%%%%%%%%%%%%%%%%%%%%%%%%%%%%%%%%%%%%%%%%%%%%%%%%%%%%%%%%%%%%%%%%%%%%%%%%%%%%%%%%%%%%%%%%%%%%%%%%%%%%%%%%%%%%%%%%%%%%%%%%%%%%%%%%%%%%%%%%%%%
    % what is a protected and unprotected group? what are the limitaitons of this categorizations in research.
    
    %%%%%%%%%%%%%%%%%%%%%%%%%%%%%%%%%%%%%%%%%%%%%%%%%%%%%%%%%%%%%%%%%%%%%%%%%%%%%%%%%%%%%%%%%%%%%%%%%%%%%%%%%%%%%%%%%%%%%%%%%%%%%%%%%%%%%%%%%%%%
    % what is collaborative filtering???????
    
    %%%%%%%%%%%%%%%%%%%%%%%%%%%%%%%%%%%%%%%%%%%%%%%%%%%%%%%%%%%%%%%%%%%%%%%%%%%%%%%%%%%%%%%%%%%%%%%%%%%%%%%%%%%%%%%%%%%%%%%%%%%%%%%%%%%%%%%%%%%%
    
    
    % SLIM learns $\langle user, item \rangle$ regression weights through optimization, minimizing a regularized loss function. Although this is not proposed in the original SLIM paper, it is possible to create a user-based version of SLIM (labeled SLIM-U in~\cite{zheng2014cslim}), which generalizes the user-based algorithm in the same way. 
    
    % \todo[inline]{check the notations}
    % Assume that there are $M$ users (a set $U$), $N$ items (a set $I$), and let us denote the associated 2-dimensional rating matrix by $R$. SLIM is designed for item ranking and therefore $R$ is typically binary. We will relax that requirement in this work, we use $u_i$ to denote user $i$ and $t_j$ to denote the item $j$. An entry, $r_{ij}$, in matrix $R$ represents the rating of $u_i$ on $t_j$.
    
    % SLIM-U predicts the ranking score $\hat{s}$ for a given user, item pair $\langle u_i, t_j \rangle$ as a weighted sum:
    
    % \begin{equation}
    %     \hat{s}_{ij} = \sum_{k \in U}{w_{ik}r_{kj}}, 
    % \end{equation}
    % where $w_{ii} = 0$ and $w_{ik} >= 0$.
    
    % Alternatively, this can be expressed as a matrix operation yielding the entire prediction matrix $\hat{S}$:    
    % \begin{equation}
    % \hat{S} = WR,
    % \end{equation}
    % where $W$ is an $M \times M$ matrix of user-user weights. For efficiency, it is very important that this matrix be sparse.
    
    % The optimal weights for SLIM-U can be derived by solving the following minimization problem:
    
    % \begin{equation}
    % \text{min}_W~\frac{1}{2}\left\Vert R - WR \right\Vert^2 + 
    %     \lambda_1 \left\Vert W \right\Vert^1 +
    %     \frac{\lambda_2}{2}\left\Vert W \right\Vert^2,   
    % \end{equation}
    % subject to $W > 0$  and $\text{diag}(W) = 0$.
    
    % The $\left\Vert W \right\Vert^1$ represents the $\ell_1$ norm and the $\left\Vert W \right\Vert^2$ term represents the $\ell_2$ norm of the $W$ matrix. These regularization terms are present to constrain the optimization to prefer sparse sets of weights. Typically, coordinate descent is used for optimization. Refer to \cite{ning2011slim} for additional details. 





    % To describe this term, we will enrich our notation further by indicating $U^+$ to be the subset of $U$ containing users in the protected class with the remaining users in the class $U^-$. Let $W_i^+$ be the set of weights for users in $U^+$ and $W_i^-$ be the corresponding set of weights for the non-protected class. Then the neighborhood balance term $b_i$ for a given user $i$ is the squared difference between the weights assigned to peers in the protected class versus the unprotected class.

    % \begin{equation}
    %     b_i = (\sum_{w^+ \in W_i^+}{w^+} - \sum_{w^- \in W_i^-}{w^-})^2
    % \end{equation}



    %%%%%%%%%%%%%%%%%%%%%%%%%%%%%%%%%%%%%%%%%%%%%%%%%%%%%%%%%%%%%%%%%%%%%%%%%%%%%%%%%%%%%%%%%%%%%%%%%%%%%%%%%%%%%%%%%%%%%%%%%%%%%%%%%%%%%%%%%%%%