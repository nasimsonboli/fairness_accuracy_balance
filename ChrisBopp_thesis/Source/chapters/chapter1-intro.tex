\begin{quote}\singlespacing“Since major decisions in the 21st century are based on big data and the proliferation of measurement, investigating the qualitative work of quantitative measurements is crucial for understanding how decisions at the ‘end’ of data analysis have already begun before data collection, during measurement, even when they are not intentional.” \citep{Pine2015Politics} \end{quote}

Human services represent the largest portion (73\%) of United States federal spending \citep{Desilver2017Fed}. Even after excluding the costly programs of Social Security and Medicare, 34\% of federal government spending is dedicated to providing critical human services such as workforce development, public health education, and child welfare. This funding is spread over many different government agencies and programs, distributed to implementing organizations where additional funds---and influence---come in from state or local governments, foundations, and individuals. 

Under the United States federalist system, decision making authority is shared by the federal, state, and local governments \citep{Salamon2002Governance} thereby decentralizing responsibility for the delivery of social services across multiple levels of government and implementing nonprofit organizations. This means that decision making is distributed within and across multiple levels. At the individual level during client meetings, at the organizational level within one or more human service organizations during the evaluation of a program, at regional- or state- level coordination task force meetings, and during federal-level legislative hearings.

The distributed nature of human services in the United States complicates the process of understanding and evaluating social change. The effectiveness of human services programs, like those that are designed to end homelessness, cannot be understood within the context of any single level---or within any single government agency or organization. Instead, the evaluation of social change must occur across multiple levels of organizations and stakeholders.

Further complicating the distributed nature of decision making in human services, individuals involved with funding, overseeing, and delivering these services also face many difficult decisions as they work to provide the best possible services for clients in complicated and difficult situations. Prioritizing one individual over another for emergency sheltering due to limited resources means that some people are literally left to sleep on the street. The decisions that human services stakeholders must make are not only urgent but can have major ramifications for human services clients.

For the individual client’s experience to be legible at higher levels of aggregation, institutions and organizations prescribe standardized measures and categories to represent the client (i.e. a vulnerability score) that, then, abstract away the specifics of the individual’s situation---such as details about the client’s support network (or lack thereof) or the effects of traumatic experiences stemming from the stigma of their situation---to derive more generalizable representations. It has been observed that human services workers question the capacity for these generalizable measures and categories to accurately represent clients’ lived experiences \citep{Schwalbe2004HS}, further complicating decision making among workers charged with designing and implementing social services.

\section{Research Overview}
In this dissertation, I examine strategies and approaches to data work in human services. More specifically, I examine \textit{what influences that data work}, including the stakeholders that have varying degrees of power, the infrastructure with its expectations for translating the world into data schemas, and the broader values and politics that shape both of these. In focusing on ``data work'', I've aligned with \citeauthor{Bossen2019Health}'s \citeyearpar{Bossen2019Health} definition, which says that data work is ``any human activity related to creating, collecting, managing, curating, analyzing, interpreting, and communicating data."

The dissertation is broken down into three empirical chapters, followed by a cross-chapter synthesis. The first chapter focuses most broadly on data work to understand the array of challenges and opportunities that are present across multiple levels of social services. The second chapter focuses specifically on the role of identifiability in human services data work to understand how individual clients are tracked and aggregated across these multiple levels of social services. And finally, the third chapter focuses on the assumptions that these stakeholders have and the ways that those assumptions shape the questions they ask and the data that they collect. Together, these chapters provide a detailed analysis of data work across multi-level computer supported cooperative work in the human services domain.

\subsection{Sociotechnical Ecosystem}
To get a broad understanding of the variety of challenges and opportunities that exist within different types of organizations in the nonprofit sector, in this first study (\autoref{ch:dbd}) I interviewed staff across several different areas of focus. This included, for example, organizations that focused on international development, fundraising, and secondary education. To get an understanding of the variety of challenges present across the sector, I also included organizations that---unlike nonprofits---aim to generate a profit while also valuing their social mission. The interviews focused on the role of data within the participant's work, the systems they use, the types of data they collect, and the challenges they encounter.

The research presented in this chapter finds that organizations are investing time, sacrificing expertise, and responding to largely external demands for data collection and reporting at the expense of the mission and operation of the organization. A major contribution of this chapter is a framework---called the cycle of disempowerment---for better understanding this sociotechnical ecosystem. The cycle captures the challenges that are encountered when attempting to use data to inform social change, which fall into three areas: erosion of autonomy, data drift, and data fragmentation.

\subsection{Aggregation, Identifiability, and Anonymity}
A topic that I observed many participants bring up as part of the previous study was the need to identify people by name or some otherwise unique identification number. In considering the cycle of disempowerment---particularly the data drift and data fragmentation components---the relationship between identifying individuals and linking data together across drifting and fragmented systems became the focus of this second study (\autoref{ch:pk}).

To examine the relationship between identification and data aggregation, the second study compares case studies of two different human service fields. Through interviews of nonprofit and government staff, the fields of homelessness and HIV/AIDS were explored from an interorganizational standpoint. Methodologically, several interviews were carried out with staff in a central nonprofit organization, and then subsequent interviews were carried out with individuals at other organizations within the central organization's network.

Through this research, I found that staff in these interorganizational networks are tasked with aggregating individually identifiable data across cities, counties, and states. However, in attempting to aggregate individually identifiable data, the database's infrastructure constrained and problematically defined the way that staff carries out their work and the way that clients are treated by the human services system. In addition to contributing these empirical findings, I also developed design provocations to inspire the investigation of more flexible data schemas in future research.

\subsection{Multi-stakeholder Questions and Assumptions}
While the previous study focused on the need to identify individuals consistently across databases, it focused most centrally on data elements like names and social security numbers. To more fully understand data collection, this third study (\autoref{ch:wkshp}) expanded the scope out to all data fields, not just identifying information. Given the focus on engaging with staff members in the first two studies, this study was specifically designed to seek out the perspectives of more diverse stakeholders, especially human services clients. 

To focus a more diverse array of stakeholders on data work, I conducted a series of workshops in which participants discussed the questions they personally ask about homelessness---the focus of this study---and the data that they have or need to answer those questions. This study design allows participants to talk to one another about the conflicts across levels, providing insight into the way that conflicts are approached cooperatively.

Through this research I found that stakeholders have assumptions around the types of questions that can be answered, the role of particular audiences, the decisions that can be made using data, and the extent to which data can be trusted. These assumptions surface entangled challenges for data work and often prioritize funders' needs over the needs of those on the ground. Through my analysis, I've found that these types of questions and their associated assumptions fall under either a system perspective or an individual perspective. By analyzing the challenges that come from each perspective, I contributed a new oligopticonic approach to data systems that may be able to negotiate between the system and individual perspectives.

\subsection{Reimagining Data Doubles as Boundary Objects}
To examine the influence of stakeholder power, politics, and infrastructural expectations, I focus on the human services client \textit{data double} and its ability to represent individuals in the evaluation of human service programs. While each chapter in the dissertation only briefly connects results to the theoretical construct of a \textit{data double}, I've found it to be particularly useful for examining the representation of nonprofit stakeholders---namely human services clients such as individuals experiencing homelessness or people living with HIV/AIDS---across this body of research. 

Acting as a digital stand-in for a person, the data double is built by merging various data streams together \citep{Haggerty2006New,Raley2013Dataveillance}. Once combined, the data double can then act---and be acted upon---as if it were the human being him/herself. By focusing on this concept, \autoref{ch:synth} contributes a new way of approaching the challenges that have been identified in the prior three chapters. By reimagining the data double as a boundary object, the data double has the potential to meet more diverse stakeholder information needs.

\section{Research Arc}
This dissertation prioritizes a deep understanding of context, challenges, and potential solutions for data work in human services. The three empirical chapters outlined above answer three synergistic research questions, which are discussed next.

Research question \#1 asks: \emph{What is the sociotechnical ecosystem around data-driven decision making in nonprofit organizations, including human service organizations?} This question was addressed mostly by the first study, though partially by all three, and focuses on better understanding the sociotechnical ecosystem of nonprofit data work. This question does not specifically focus on human services or the human services data double but instead is meant to understand the broader context that the nonprofit data double operates within. Research question \#1 aims to understand the opportunities and challenges, as well as the context around why these exist and what the implications are of each. 

Research question \#2 asks: \emph{How do data doubles come to exist through the aggregation of data among networked members of this ecosystem?} This research question leverages and extends the contextual understanding developed as part of research question \#1 to focus more specifically on the ways that individually identifiable data is tracked across this ecosystem. In theory, such identifiability would be used to merge data sets to construct the data double. To answer this research question, I draw upon \autoref{ch:pk} to understand the challenges of carrying out identifiable data aggregation.

Research question \#3 asks: \emph{How are data doubles shaped and used by different stakeholders across a federalist system of human service implementations?} This research question moves beyond the focus of aggregation in research question \#2 to look more holistically at the data double. In terms of shaping the data double, the assumptions that stakeholders have can speak to the underlying values and politics that influence what data is or is not collected and is thereby available to be incorporated into the data double. The use of the data double depends on the questions that stakeholders have because the data double is only invoked when a decision must be made about the individual person it represents. Given the breadth and complexity of this question, I draw on all three empirical chapters to understand the shaping and use of data doubles.

\section{Research Contributions}
By focusing on the context, challenges, and potential solutions for data work in human services, this dissertation as a whole contributes an empirical understanding of factors that shape human services data doubles and the ways that they are used or not used. This research also contributes to the critical data studies literature on the shaping and performance of data doubles, particularly under the unique sociotechnical conditions of human services work. In addition, this research contributes to an emergent body of scholarship in philanthropic studies about the role of data in the nonprofit sector, specifically the legibility of clients, and tensions among rationalized expert-driven models and the experiences of local individuals, such as nonprofit staff and clients.

Through the inclusion of diverse stakeholder populations in this research---particularly clients, whose voices are often not given equal value as those of professional staff or policymakers \citep{Bopp2019Voices}, and nonprofit human service organizations, who are frequently disempowered by the demands of data \citep{Bopp2017DbD}---this research has catalyzed dialog among stakeholders. This dialog has laid an essential foundation for the co-design of data infrastructures that more equitably balance the needs and values of multiple stakeholders, which contributes to the broader research agenda of informing the use of big data for the benefit of society.