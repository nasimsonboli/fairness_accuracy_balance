\section{Summary of Contributions}
This research illustrates that in order to support the kinds of innovative uses of big data that many desire in the social sector, a number of challenges must be addressed. What ties these challenges together is a need to listen to more diverse stakeholders in the design and implementation of data systems. These systems must be designed to meet the needs of clients, staff, and the community.

\hyperref[ch:dbd]{Chapter 3} contributed an empirical understanding of the way that staff members carry out data work in a broad array of mission-driven organizations. Grounded in that empirical understanding, the cycle of disempowerment framework was developed to bring together three negative consequences of current data work: erosion of autonomy, data drift, and data fragmentation.

Through an exploration of data work in a network of organizations focused on HIV/AIDS and homelessness, \autoref{ch:pk} contributed a design space for understanding the use of client identifiers along axes of anonymity and uniqueness. Relying on this design space, four classes of sociotechnical human services work afforded by the primary key were described. The challenges introduced by the primary key were understood by examining the way that the database's technical abstractions influenced the nature of human services work. Finally, this chapter contributed design provocations for thinking through these challenges.

Finally, \autoref{ch:wkshp} contributed an empirical understanding of the assumptions that stakeholders have regarding system and individual level questions and outcomes, as well as a variety of challenges that result from these assumptions. Grounded in this empirical work, this chapter contributed an understanding of the relationship between the system/individual perspectives and the concept of an oligopticon. Leveraging this concept, design requirements for meeting more diverse informational needs through an oligopticonic approach were also developed.

Based on these empirical results, \autoref{ch:synth} advocated for the use of data doubles as boundary objects in order to more directly involve staff members and clients---who are more directly involved in the day-to-day work of nonprofit human services programs---in the design of data representations. While many powerful external actors like funders have control right now, that control needs to be shared with those on the ground. A more equal distribution of power can support the complex needs for things like varying degrees of anonymity and uniqueness in database identifiers and oligopticonic data systems. There is also reason to believe that outside of this human services domain, other instances of data doubles should be reimagined as boundary objects to be more inclusive of the perspectives of the individuals they represent.

\section{Research Questions}

The research contributions outlined above address all three of the research questions that were posed in \autoref{ch:intro}. I will now discuss these contributions in relation to the research question that they addressed.

Research question \#1 asks: \emph{What is the sociotechnical ecosystem around data-driven decision making in nonprofit organizations, including human service organizations?} We see that the sociotechnical ecosystem is characterized by the cycle of disempowerment described in \autoref{ch:dbd} where human services work is not being driven by data but is instead being driven by the demands of data. Organizations are investing time, sacrificing expertise, and responding to largely external demands for data collection and reporting, often at the expense of the mission and operation of the organization. The sociotechnical ecosystem is further shaped by the infrastructural requirements for consistent identifiability---as described in \autoref{ch:pk}---which push up against the diverse needs for anonymity and uniqueness in identifiers. Finally, \autoref{ch:wkshp} shows that the sociotechnical ecosystem is characterized by the prioritization of the system perspective assumptions made by stakeholders which influence the questions that get asked and the data that is collected.

Research question \#2 asks: \emph{How do data doubles come to exist through the aggregation of data among networked members of this ecosystem?} Data about the individual human services client is aggregated across networks of organizations. As described in \autoref{ch:pk}, this aggregated data is used to count, verify, track, and refer clients across interorganizational networks. Attempts to construct data doubles in central locations to serve institutional agendas are not only of questionable value to stakeholders across the multiple levels of human services, but the prioritization of data aggregation introduces additional challenges as a result of the primary key technical requirements. The technical infrastructure for aggregation constrains and problematically defines the way that staff carry out their work, as well as the way that clients are treated by the human services system.

Finally, research question \#3 asks: \emph{How are data doubles shaped and used by different stakeholders across a federalist system of human service implementations?} Data doubles are \emph{shaped} by assumptions that stakeholders have across the ecosystem of human services. More specifically---as described in \autoref{ch:wkshp}---these data doubles are often \emph{shaped} by assumptions about audience, decision making, and knowability, especially the assumptions that are prioritized by the system perspective. Data doubles are also \emph{shaped} by the infrastructure's requirements for consistent identifiability as described in \autoref{ch:pk}, as well as components of the cycle of disempowerment as described in \autoref{ch:dbd}.

As described in \autoref{ch:wkshp}, these data doubles are \emph{used} by decision makers for efficiently using limited taxpayer funding and to convince audiences of the need to address the problem. These data doubles are also \emph{used} for counting, verifying, tracking, and referring clients (\autoref{ch:pk}). Finally, these data doubles are \emph{used} for responding to largely external demands for data (\autoref{ch:dbd}). Together this research has found that instead of driving towards institutionally oriented data doubles, they should instead be reimagined as being cross-stakeholder oriented---more specifically as a boundary object. However, such a change requires reimagining assumptions while up against the expectations of both the cycle of disempowerment and the technical infrastructure.

\section{Limitations \& Future Work}
While this research has begun to touch the surface on disrupting the cycle of disempowerment, more work needs to be done. Based on the methods employed in this research, there is a need to better understand the extent to which other geographic areas, organizations and individuals are experiencing similar challenges. This research has gathered rich qualitative data which (as described throughout this dissertation) is essential data to help us understand the complexities of the challenges in this space. However, future research should be conducted to better understand the extent to which these challenges are occurring elsewhere. Anecdotally, nonprofit professionals that I have spoken to about this research have the opinion that these challenges are occurring across the sector, however this would need to be validated empirically.

In \autoref{ch:synth}, I discussed the need to reimagine human services data doubles as boundary objects. There is a significant amount of future work needed to better understand what would be involved in such an endeavor. As suggested in that chapter, there is a question as to whether or not it would be even possible to create a new boundary object in this context given the power dynamics between funders and other stakeholders. In order to investigate this, there are three related but distinct areas of research that should be pursued in the future to build off of this research: developing flexible oligopticonic data structures, integrating qualitative data, and designing for bottom-up feedback and control.

\subsection{Flexible Oligopticonic Data Structures}
This first area takes a much more technical approach than any of the research in this dissertation by focusing on the implementation of databases and data structures. While I have outlined what nonprofit human services data needs are from a sociotechnical perspective, there remains a number of questions of how to support this technically. Databases---especially relational databases---thrive on an architecture of centralized control. Schemas must be rigid and defined up front, however what this research calls for is more flexibility.

To implement this flexibility, I see promise in NoSQL databases which eschew the traditional tabular data structures used by relational databases and instead employ a more flexible key-value structure. These types of databases have seen significant software development over the past decade to facilitate their support of big data and Web 2.0 applications. Investigating the possibility of designing new information systems for nonprofit human services organizations that leverage this type of database technology could address some of the architectural needs outlined here.

\subsection{Integration of Qualitative Data}
The need for integrating qualitative data into nonprofit work has been identified by numerous scholars over the years (e.g. \citet{LeDantec2008Trenches,Verma2016DrillDown}), however this is an area that needs more focused attention. While this has been identified as a significant need, we know very little about 1) what it would take from a design perspective to make this type of data actionable from a system perspective, 2) if this data would need to be somehow summarized even from the individual perspective, 3) how structured or unstructured this data should be, and 4) how this data should be best input and/or accessed through digital technology as opposed to a paper format.

In pursuing this direction of research, it would be beneficial to look towards other domains such as healthcare and education to see if there are insights that can be transferred to the human services domain. Electronic medical record systems and student records systems may have implemented qualitative data support mechanisms that can be studied and transferred to systems like the Homeless Management Information System.

\subsection{Bottom-Up Feedback and Control}
The two areas of research discussed so far have a more technical focus---which is needed to provide a more well-rounded perspective to the research outlined in this dissertation---there is also more work needed from a social perspective. If more diverse stakeholders are to be able to create a boundary object, they must be supported in doing so. This last area of research would explore that aspect further to understand what would be involved in supporting that work. It is unknown if stakeholders like funders would need to be convinced to be more flexible, or if such work can take place without their permission. It may be possible that designing the ``Yelp for homeless shelters'' that was discussed in \autoref{ch:wkshp} would begin the process of constructing a new boundary object without needing to get approval from those in power.

Along these same lines, it would also be useful to imagine designing systems that allow individual clients to retain control over their data rather than leaving it in the hands of others to manipulate for their own needs. Such research would allow us to better understand the power distribution among these multi-stakeholder networks, as well as what the impact of restructuring power might facilitate.

In summary, this research has found that there are a number of challenges in using big data in the nonprofit context, but by listening to more diverse stakeholders in the design and implementation of data systems we can conceivably come closer to realizing their promise of more fair and equitable treatment of people throughout the world.