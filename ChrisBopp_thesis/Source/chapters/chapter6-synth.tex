Based on the findings from each of the three empirical chapters of the dissertation (\autoref{ch:dbd} through \autoref{ch:wkshp}), this chapter will synthesize across all studies and relate that synthesis to the literature covered in \autoref{ch:lit}. This chapter will begin by discussing the most central concept---data doubles---both conceptually and empirically. Following this discussion, I propose reimagining the data double as a boundary object to better address stakeholder needs.

\section{Data Doubles}
Conceptually, a data double refers to the set of data that is collected about a person. This set of information is collected and combined across different sources. For example, information about your schedule from your calendar app might be combined with information about your physical location from your navigation app. In combining these partial data sets, a data double can be created that better understands who you are as a person than if the data sets were left separate. Once the data double is created, it can stand in your place the same way a stunt double might stand-in for a movie star---acting as if it were the actual person even though it is not.

\paragraph{Information Flows}
Scholars that have analytically leveraged the concept of the data double have highlighted several important characteristics. The first is that it is comprised of ``information flows" \citep{Haggerty2006New} which carry ``flecks of identity" \citep{Fuller2005Media}. The term ``information flows" refers to the capacity of computer networks to bring data together from diverse information systems so that more accurate representations can be constructed. This argument is supported by the belief that big data can offer better insight than smaller data sets \citep{Boyd2012Critical}.

Using the networked capacity of information systems to establish information flows, flecks of identity can be transferred from one database to another. While each fleck of identity that is associated with a living human being does not mean much---it is only a fleck after all---these flecks can be combined with one another to become more substantial and meaningful.

\paragraph{Central Locations}
The second important characteristic of a data double is that networks are used to transport flecks of identity to centralized locations, most often to places of power and influence. A large e-commerce platform could, for example, combine product searches with e-book reading and streaming TV viewing habits to better understand topics that you are interested in. In pulling these flecks of identity together, the platform can use their powerful position to target ads or modify search results to encourage you to purchases products that have higher profit margins for the company.

\paragraph{Institutional Agendas}
The e-commerce platform's position of power allows them to use the consumer's data double to serve their interests, which is the third important characteristic of a data double. \citet{Haggerty2006New} say that data doubles are used ``in ways that serve institutional agendas.” Since these centrally-located institutions are often the ones who are also defining how data should be collected, transmitted, and stored in the first place, they are in the best position to extract the maximum amount of value from it.

\paragraph{Performative}
The fourth characteristic of the data double is that it is performative. The abstract data that the institution relies on can be used to target an actual living human being \citep{Raley2013Dataveillance}. While the individual consumer in the e-commerce example may only be impacted by paying a dollar or two more, the implication of this characteristic in the human services context carries much higher stakes. In using a data double to determine eligibility for a homelessness program, such an institutional decision can mean that someone has to sleep on the street instead of being safe inside.

\subsection{Data Doubles in this Research}

\subsubsection{Information Flows}
In the research discussed in \autoref{ch:dbd} through \autoref{ch:wkshp}, we see that there is a lot of time and energy being invested in creating data doubles for use in human services work. In the interorganizational networks discussed in \autoref{ch:pk}, information flows from the organization's local databases into systems like the region's Homeless Management Information System or the county's HIV/AIDS database. In \autoref{ch:wkshp}, participants described building a ``bridge between'' information systems including hospitals and jails.

Due to internal data fragmentation as discussed in \autoref{ch:dbd}, nonprofit staff often must first do their own merging of information flows to pass on that data to higher levels of aggregation. While some other uses of the data double concept have observed that information flows are automated, in the context of nonprofit human services work the process is very manual. Once assembled internally using a tool like Microsoft Excel, data sets are often emailed out so that the flecks of identity that individual organizations hold about individual clients can be transported over networks and centrally combined.

\subsubsection{Central Locations}
The first two characteristics of the data double---that flecks of identity are transferred via information flows, and that these flows are centrally combined in places of power---are made possible in human services through the primary key as described in \autoref{ch:pk}. Ideally, the infrastructural abstraction of the primary key would work to enforce the consistent identifiability of individuals so that data doubles can be accurately assembled across information flows. As described in \autoref{ch:pk} however, due to the need for varying levels of anonymity and uniqueness across contexts, the primary key does not work as imagined.

Given the power that these central locations have, those providing the information flows often have no choice but to comply. As we've seen in the cycle of disempowerment in \autoref{ch:dbd}, organizations experience an erosion of autonomy. Even if organizations do not want to participate in providing the information flows that will be used to form data doubles, they often do not have a choice because data collection and reporting is often a condition of obtaining funding. As described in \autoref{ch:pk}, there is a wicked irony in that the local organizations cannot use aggregated data to help in decision making. These local organizations do the majority of the work of collecting data, but it is only the funders who can use the data doubles once they are consistently assembled across information flows.

\subsubsection{Institutional Agendas}
The power imbalance between local providers and the centralized data aggregators illustrates the third characteristic of the data double, that it is designed to serve institutional agendas. As discussed in \autoref{ch:wkshp}, while stakeholder see value in the egocentric perspective, it is the sociocentric perspective that often takes priority. The sociocentric perspective is most closely aligned with the institutional agenda by prioritizing mandated counts and scores that can be used to effectively allocate limited financial resources.

In using counts and scores, the sociocentric perspective is oriented towards making homelessness legible enough to its institutional audience. In \autoref{ch:pk} we saw that this manifested as the need to count clients so that community-level supply and demand for services could be managed, and to ensure that only individuals deemed eligible for services received them. This sociocentric perspective serves the institutional agenda of minimizing costs to taxpayers and using resources effectively.

\subsubsection{Performative}
In the process of serving institutional agendas, data doubles that can target an individual's life are constructed. The quote from \autoref{ch:pk} speaks directly to the targeting that can be accomplished through the creation of the data double:

\begin{quote}\singlespacing ``We know [our high-use clients] by name now. We can alphabetize them, we can talk to them, you know, we can start convening committees to work around, you know, what are we going to do about it." \end{quote}

The data double's targeting of an individual's life has very significant consequences. For example, as discussed in \autoref{ch:wkshp}, a person experiencing homelessness was bypassed for housing services because he has not interacted with the police as much as others have. The data double that was designed for the sociocentric perspective thereby values reducing costs to taxpayers and, as a result, it leveraged to prioritize housing services for those who have interacted with more costly services, like the criminal justice and healthcare systems. This person's data double has been used to target him, and it has had a profound influence on his life.

\subsection{Identified Challenges in Constructing Data Doubles}
As has been discussed throughout this dissertation and summarized above, there many challenges that are encountered in constructing data doubles in the human services context. In bringing together information flows, data fragmentation makes the manual assembly of data sets even more difficult and time-consuming. Flecks of identity cannot be consistently merged into a data double given the need for privacy. The sociocentric perspective that is prioritized when collecting data means the data double is less useful for day-to-day decision making. While there may be opportunities for resolving these issues from both a social and technical perspective, it is important to first ask if pursuing data doubles---as currently defined---is the right direction in the first place.

In the remaining sections of this chapter I will argue that pursuing data doubles as currently defined above is not the best direction forward. Given the data double's institutional orientation and its need for consistent identifiability, I argue that it is presently designed to prioritize the needs of the institutions over the privacy of individuals. What I believe these findings suggest is the need to redefine the data double as a boundary object. In doing so, the data double can better meet the needs of stakeholders.

\section{Data Doubles as Boundary Objects}
So far I have argued that in this human services context, we see not only the presence of data doubles, but we also see attempts to realize more complete and reliable data doubles in the future. I will now argue that data doubles---as currently defined and operationalized---are \textit{not boundary objects}, but are instead institutionally oriented infrastructure that does not meet the needs of all stakeholders. I will subsequently argue that it is possible to meet the needs of more stakeholders by reimagining the data double \textit{as a boundary object} which has the capacity to disrupt the ill-suited infrastructure. To my knowledge other researchers have not discussed data doubles and their relationship---either as aligned or out of alignment---with boundary objects. Based on my analysis, I would also argue that other applications of the data double have not been boundary objects.

According to \citet{Star2010Boundary}, the creation of the boundary object concept ``was motivated by a desire to analyze the nature of cooperative work in the absence of consensus.'' As we have seen in the previous three empirical chapters---which echoes the findings of \citet{Star1989Ecology}---consensus is not needed to carry out cooperative work. What we have seen in the human services context is that, despite a number of challenges, there is in fact good work being done by multiple stakeholders to address homelessness and to reduce the spread of HIV. Despite a lack of consensus, it is still possible for different stakeholders to speak to different audiences and carry out a different set of tasks to work cooperatively. The key to satisfying each stakeholder's different informational needs is the boundary object, which can ``inhabit several intersecting social worlds'' \citep{Star1989Ecology}.

To be considered a boundary object, the single object must support interpretive flexibility which refers to the way that it may be interpreted differently by stakeholders. For example, \citet{Star1989Ecology} explain that a road map may be used by campers to find campsites while the same map may be used by scientists to find animal habitats. The same boundary object (the map) serves the information needs of both campers and scientists without the need for consensus.

If the data double were to be reimagined as a boundary object, there is reason to believe that it would meet the information needs of funders, nonprofit staff, and clients without the need for consensus. The biggest shift in reimagining the data double as a boundary object would be a move from a centralized and institutionally oriented object to one that meets the information needs of the institution \textit{as well as} other stakeholders. While the data double has largely been used for accountability---tracking nonprofit work and individuals to show that funds have been spent appropriately---a data double reimagined as a boundary object could also reasonably address day-to-day operational decision making.

Boundary objects are ``subject to reflection and local tailoring" \citep{Star2010Boundary}, which suggests that data doubles reimagined as boundary objects would support the capacity for clients and front-line staff to make adaptations that are useful to them. This would be accomplished by alternating the data double between its identity as a common, shared object and a locally-tailored object. When the object is at the local level, the stakeholder would have the ability to reshape the object before returning it to its shared status. Following this back-and-forth tacking, the object can then scale up and become standardized, turning into infrastructure and standards \citep{Star2010Boundary}. In essence, what this calls for is a multi-stakeholder approach to iteratively redefine the data double in a way that addresses each stakeholders' needs. Once the data double has been redefined several times, only then can it become standardized and integrated into systems like the Homeless Management Information System.

Over time, it is inevitable that the boundary object will begin to be less able to support interpretive flexibility. The formal representations used in infrastructural systems like the Homeless Management Information System will once again be missing important data points that stakeholders rely upon. There will be a variety of needed data points---what \citet{Star2010Boundary} refers to as ``residual categories''---that do not fit on the official forms. It is at this point that a new boundary object would be formed, and the cycle would begin again.

There are two networks discussed in this dissertation that a data double that operates as a boundary object would be more capable of circulating within. These networks are the US federalist system---or ``scales of influence and accountability” as discussed by \citet{LeDantec2008Trenches}---and sites of inquiry as suggested by the oligopticon. Data doubles re-defined as boundary objects would be capable of circulating within these networks to seek, not consensus, but a way to meet more information needs. The data double could be restructured by the federal government by adapting it to their local needs and then returning it to its global status to be subsequently modified by people experiencing homelessness. Within the oligopticon, the boundary object could be redefined by employers who hire someone experiencing homelessness and also redefined by the individual themselves. No longer institutionally oriented, as a boundary object the data double is more open to local adaptations.

\section{Implications for Human Services Organizations}
While the discussion above is largely conceptual, I'd like to translate what this transition to a boundary object might mean practically for human services organizations and other stakeholders within that context. The entry point into a boundary object is through residual categories, meaning the items that don't fit into current forms. The text that might go into the ``other field'' or get written in the margins of forms comprises the residual categories that are missing from the current infrastructure. My research has identified some of these residual categories (e.g. capability to do laundry at one's house or the emotions that one experiences being on the street), but there are undoubtedly many more that I have not identified.

As residual categories are identified, stakeholders must create new boundary objects that support those information needs. This is much easier said than done. The task of creating a new boundary object may in fact prove to be impossible given present power dynamics. As discussed extensively in \autoref{ch:dbd}, the cycle of disempowerment means that the individuals who are in the best place to advocate for a new boundary object are exactly the same individuals who have little to no power to accomplish such a feat. It is unclear based on this research if creating a new boundary object is feasible. However, now that we have a better idea of what is needed from a data infrastructure standpoint, future work can focus on ways to support that task. It may be necessary to lobby decision-makers for the capacity to establish new boundary objects. Or, it may require a more grassroots effort to encourage those most familiar with residual categories to step up and put a new boundary object in place. After all, if consensus is not needed, perhaps there is enough space in this cooperative work to advocate for the information needs of more diverse stakeholders.