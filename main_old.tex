% \documentclass{article}
\documentclass[manuscript,screen,review]{acmart}

\usepackage[utf8]{inputenc}
\usepackage{graphicx}
\usepackage{caption}
\usepackage{subcaption}
\usepackage{bbm}
\usepackage{color}
\usepackage{colortbl}
\usepackage{hyperref}
\usepackage{multirow}
\usepackage{enumitem}
\usepackage{algorithm}
\usepackage{algorithmic}
\usepackage{enumerate}
\usepackage{amsmath}
\usepackage{booktabs}
\usepackage{multirow}
\usepackage{float}
\usepackage{mathrsfs} % needed for mathscr
\usepackage{color}
\usepackage{balance}
\usepackage[bottom]{footmisc}
\renewcommand{\algorithmicrequire}{\textbf{Input:}}
\renewcommand{\algorithmicensure}{\textbf{Output:}}
\usepackage[para,online,flushleft]{threeparttable}
\usepackage{verbatim}
\usepackage{dsfont}
\usepackage{todonotes}
\usepackage{tikz}
\usetikzlibrary{shapes,arrows,spy,positioning,decorations,decorations.pathreplacing}
\newcommand{\libauto}{\texttt{librec-auto}}


\title{FARS: Fairness-Aware Recommendation Algorithms}
\author{Nasim Sonboli}

\begin{document}
\maketitle

\section{Introduction}
saying the problem i am solving (describing the problem) through an example or two. then talking about the proposed solutions and contribitions.
\paragraph{\textbf{Summary of contributions}}

we propose an inprocessing algorithm
we propose two re-ranking algorithms
pfar/far
ofair
we propose the SRUFF framework
contributions to the library librec-auto

\section{Algorithmic Fairness}

% motivation
After a period of substantially fast development in different aspects of digital systems, and with more and more decisions being delegated to algorithms, the society has begun to realize these systems that were intended to assist people in different tasks have ethical issues and can cause harm to individuals and the society.


Take as a real-world example the following study: \cite{1Muhammad2019facebookads} studied the distribution of ads on Facebook to understand potentially discriminatory impact in the visibility of different kinds of ads. They found that even when an advertiser wishes to have fair distribution of their ad, for example to ensure that an ad for a job opening is seen by people of all genders, the combination of relevance optimization and market dynamics results in disparate distribution of ads across racial and gender lines. 
% \cite{barocas2016big} has caution that algorithms can introduce new biases in systems or perpetuate existing ones.

Discrimination caused by algorithms that are trained on biased data or by lack of a good design, propagation of bias \cite{barocas2016big}, marginalization of minority groups in the society, inflation in the polarization of the society that can be caused by tight filter bubbles due to massive filter bubbles, and etc. are among the harms that algorithms can potentially cause. These problems have gained the attention of a multidisciplinary community from computer scientists, social scientists and legal scholars. Thus as a response, recent research has shifted from design of algorithms that pursue purely optimal outcomes with respect to an objective function into ones that also consider social impacts such as fairness.

\subsection{Fairness in Machine Learning}
% fairness in ML
Fairness, bias and discrimination are topics of considerable research interest in the recent years~\cite{pedreshi2008discrimination,fairness,bozdag_bias_2013}.

% what fairness means in machine learning? how fairness is detected in machine learning? What are the solutions presented?
Much of the work in algorithmic fairness has been focused on classification methods with a myriad of definitions that has been proposed. %\todo{cite}.
The key definition are explained in \cite{mitchell2021algorithmic}.

One of the main divisions in fairness definitions comes from the way we assess and evaluate fairness and whether this evaluation is individual based or group based.

\subsubsection{Group Fairness}
%group fairness
In group fairness based definitions, a model’s treatment of two or more groups with respect to a sensitive attribute (e.g. gender, race, ethnicity, etc.) is compared. In this method, the protected group(s) is designated with respect to a previously defined sensitive attribute and should be protected against discrimination. The sensitive attribute definition is usually rooted in anti-discrimination laws\cite{barocas2016big}.This notion of fairness tries to ensure that algorithms don't impact the members of the protected group more adversely and disproportionately. Group-based methods of fairness has helped build the most prevalent structures to achieve and assess fairness ~\cite{zemel2013learning,kamishima2012fairness,kamiran2010discrimination,zhang2017anti}.

\paragraph{Statistical Parity}
% Statistical Parity
% This notion has become the most prevalent structure, but even there researchers have shown tension between different definitions \todo{cite}.
The notion of Statistical or Demographic Parity requires that two groups with a different sensitive group have the same chance of getting a positive result. This notion has been discussed under different names of avoiding disparate impact \cite{Feldman2015}, independence \cite{barocas2018fairness} and anti-classification \cite{corbett2018measure}. This measure is used to ensure a ``fair'' representation of different groups in different tasks such as ranking\cite{singh2018fairness,zehlike2017fa,yang2017measuring}, and recommendation\cite{mehtora2018towards,ekstrand2018exploring}.
    
\paragraph{Performance Parity}
    % Performance Parity
Performance Parity is another category of group fairness that requires equal error rates for different groups. Equality of opportunity or equality of true positive rates \cite{hardt2016equality} that requires the positive classification rates to be independent of the protected attribute given the true label, equality of both true positive and false positive rates which is known as equalized odds \cite{hardt2016equality}, equality of mis-classification rates (e.g. equality of false negative rates aka lack of disparate mistreatment\cite{zafar2017fairness}) and equality of positive predictive values aka calibration, belong to this category. 
    Performance Parity has also been studied as error parity in recommendation algorithms\cite{ekstrand2018all,yao_huang_fatml-2017}.

\subsubsection{Individual Fairness}
%individual fairness
Dwork et al. \cite{Dwork2012individual} observed that the demographic parity requirements can be met when qualified candidates from one groups and random candidates from the other groups were chosen. Thus, satisfying certain group based fairness notions might degrade fairness for individuals in a group. Therefore fairness might be a requirement on an individual level.
Dwork et al. \cite{Dwork2012individual} introduces the concept of individual fairness \cite{Dwork2012individual} which posits that similar individuals with respect to the task at hand should be treated similarly or in other words should have similar probabilities of positive classification outcomes. One of the limitations of this method is the choice of similarity metric to compare individuals and whether this metric is unbiased. This method also doesn't place any requirement on the treatment of dissimilar individuals.
Another individual fairness definition was proposed by \cite{pmlr-v70-kearns17a} for the problem of candidate set selection from diverse incomparable source sets. Choosing candidates for a research position from a diverse research communities with uncomparable research metrics (e.g. citation rates are different in different research communities) is an example of this problem. Meritocratic fairness requires that less qualified candidates are probabilistically almost never chosen over the more qualified candidates.

\subsubsection{Harm}
% harm!
Independent to group and individual fairness, Crawford \cite{crawford2017trouble} defines two new fairness definitions which are connected to the harm caused by unfairness: (a) distributional harm that is cause by an inequitable distribution of a resource or opportunities, and (b) representational harm, where a system doesn't have an accurate representation of the society or where it systematically misrepresents certain groups.

\subsubsection{Anti-classification \& Anti-subordination}
The motivation of fairness constructs also categorizes fairness definitions into two groups: anti-classification and anti-subordination. U.S. anti-discrimination law is rooted in anti-classification ideas which requires that the influence of a protected group should not play a role in the decision making process. This notion can also be called as disparate treatment. The main idea of anti-subordination is to actively work to reverse the effects of historical discriminatory patterns in the decision making processes \cite{barocas2016big}.

Although these motivations are often very clear, their proper application is still vague \cite{xiang2019legal}. It is also worth mentioning that it has been shown that satisfying different fairness notions at the same time is mathematically impossible and infeasible \cite{Kleinberg:InherentTrade,chouldechova2017fair}. Due to the competing and sometimes conflicting goals of different fairness definitions, different needs of various stakeholders involved, etc. we cannot have a system that is universally "fair". Thus, it is essential to pick a fairness notion that serves best in the specific context of the target application. 

\subsection{Fairness in Recommender Systems}
% what are recommender systems?
Recommender systems are one of the most pervasive applications of machine learning in industry. They play a pivotal role in connecting users to relevant items or content throughout the web while not only users rely heavily on them but also content producers, sellers or information providers.

Consider a recommender system suggesting job opportunities to job seekers. Discriminatory recommendations in this system could mean that men and women with similar qualifications don't get recommendations of jobs with similar rank and salary. Or when women get similar recommendations just because of their demographic information not because of their qualifications. The system would therefore need to defend against biases in recommendation output, even biases that arise due to behavioral differences: for example, male users might be more likely to click optimistically on high-paying jobs.

% How this problem extends to other areas of machine learning such as recommendation system?
Traditionally the focus of recommendation algorithms have been on accuracy and it was known that it is tied strongly with user satisfaction. Later on, the focus of these systems changed to \emph{beyond accuracy} methods such as diversity, coverage, novelty, serendipity. This change of focus was supported by the literature that showed these properties in recommendation lists of users increase their overall satisfaction.
%  \todo{cite}
In recent years, aligned with the change of focus on these beyond-accuracy and socially sensitive properties in machine learning, the social aspects of these algorithms has come to the fore.

% to the injustice they have caused for the society \todo{cite}.
Therefore achieving fairness in recommendation algorithms has gotten more attention. However, the goal of fairness isn't completely new in the recommender system literature. Alleviating the problem of popularity bias in recommendations \cite{popbias2018} and ensuring equality or equity in long-tail recommendations \cite{ferraro2019} can be thought of achieving fairness for content providers in the systems. Group recommendation also tries to recommend items to users while considering and treating all the members of the group fairly \cite{kaya2020}. However, the goal of fairness in recommendation goes beyond one stakeholder and is not bounded to the previously mentioned problems. Rather it focuses on the aspects that are socially sensitive such as discrimination against sensitive-groups, under-representation of sensitive groups and preventing biases from creeping into these systems.


\subsubsection{Challenges of Fairness in Machine Learning}
Recommender systems have their unique challenges for investigating the fairness concepts and the methods that have been developed in other machine learning literature is not fully applicable in recommender systems. The role of personalization and multistakeholder nature of recommender systems add major additional complications to the problem of fairness in recommendations.
% ranking the outputs and the context add additional complications to the problem of fairness in recommendations.
    
\paragraph{Personalization}
% personalization
There is a tension between the goals of personalization and fairness \cite{modani2017fairness}. On the one hand, the goal of personalization is to find the best item(s) for each user while this list could be different for the users. On the other hand, fairness for providers means to give them equal visibility. So, one could simply divide the recommendation opportunities equally among providers. In this case personalization for consumers and fairness for providers are conflicting goals and achieving one means sacrificing the other one.

Additionally, recommendations become homogeneous over time in iterative environments \cite{Chaney2018} causing multiple issues like deteriorating popularity bias in a system. This issue both causes less visibility for less popular or marginalized item providers (provider unfairness), and can be unfair to marginalized consumers as popularity bias in the input data can cause the preference of 95\% the users be overshadowed by only the preference of 5\% of users who have rated mostly popular items (consumer unfairness) \cite{Eskandanian2019power}.


% Personalization interferes with the goal of fairness. This is mainly because fairness for all means that all the users in a system should receive the same recommendations while the goal of personalization is to find the best item for each user which could be different for different users.

%Besides maximizing the lenders' interests, we also consider the allocation of recommendation opportunities across the borrower side. The concepts of personalization and fairness are conflicting to some extent \cite{modani2017fairness}. On the one hand, the main goal of personalization is to break the absolute fairness so that the recommended loans can best match the lenders' interests and needs. On the other hand, to obtain the ideal fairness, one could simply divide the recommendation opportunities equally to each region.
    
\paragraph{Multistakeholder aspects}
% multistakeholderness
Recommender systems exist to facilitate transactions between consumers, content providers. Thus, many recommendation applications involve multiple stakeholders and therefore may give rise to fairness issues for more than one group of participants~\cite{burke_multisided_2017}.
As an example, Uber Eats is a multistakeholder setting, where consumers are users who order food, providers are restaurants who provide the food, Uber itself is the system, and drivers are the other stakeholders involved as well. Fairness concerns of any or all of these entities are important and necessary.
Consumer fairness, is concerned with fair and equitable treatment of all the users in the system regardless of their membership to any protected group. For example ensuring that all the subgroups of users are receiving quality recommendations not only certain groups. Provider fairness is concerned with a fair treatment of content providers or content creators. For example, by making sure they are represented in the recommendations fairly and have equal opportunities to benefit from the system. Subject fairness is concerned with fair treatment of the content, people or entities in a system. For example, ensuring that recommendations do not systematically under-represented specific segments of the society or certain content.

Therefore, recommendation fairness is not one problem. Multiple stakeholders might be involved and although they might all seek fairness, their goal of fairness might be different (due to different definitions of fairness), competing or conflicting.

For the previous two challenges, take the job recommendation scenario as an example. Since a recommender system is often in the position of facilitating a transaction between parties, such as job seeker and prospective employer. Defending biases towards both parties may be important. For example, at the same time that a job recommender system is ensuring that male and female users to get recommendations with similar salary distributions, it might also need to ensure that jobs at minority-owned businesses are being recommended to the most desirable job candidates at the same rate as jobs at majority-owned businesses such as white-owned businesses.

Due to issues as such, researchers in recommender systems have begun to seek ways to ensure fairness in the results that such systems produce for multiple parties (stakeholders).

Defeating such biases becomes more challenging when a system's main goal is personalization. Even in the example before, some users might prefer a somewhat lower-paying job if it had other advantages: such as a shorter commute time, or better benefits. 
Personalization is so important that if a job seeker does not find the system's recommendations valuable, he or she may ignore the fair aspect of the system and may migrate to a competing platform. The same is true of job providers; a company may choose other platforms on which to promote its job openings if a given site does not present its job ads as recommendations or does not deliver acceptable candidates. Therefore the fairness goal has some tensions with the personalization goal.

% how fairness is detected in recsys? and this ...
\paragraph{Other Challenges}

As mentioned previously, we cannot achieve a universally fair system, therefore for each problem, it's essential that we consider the target stakeholders, the definition of harm or unfairness and the specific metrics for measuring harm or integrating in the system to avoid harm. These elements are important in order to define, integrate and assess fairness in recommendation algorithms.
% Besides the previously mentioned challenges, 

Lack of appropriate data to study fairness goals is another challenge with which we have to deal. We might need sensitive attributes (e.g. gender, race, etc.) for the stakeholder entities but sharing and using this data might have privacy, legal or ethical concerns. Some of solutions that were used for this problem were using crowd sourcing\cite{biega2020overview}, or professional annotation, integrating different datasets, using inference methods to impute demographic information, generating synthetic datasets \cite{burke2018synthetic}, or training algorithms without demographics\cite{Kallus2020Assessing}. But, all of the previous solutions bring their specific limitations to the method. 

One of the other key challenges in this area is the domain-specificity of recommendation environments. The utilities that are delivered to each class of stakeholder are highly dependent on the type of item being recommended, the social function of the platform, and the interactions that it enables. It is therefore difficult to find appropriate data sets for experimentation and challenging to generalize across recommendation scenarios.

\subsection{Fairness for Multi-stakeholders}
% fairness solutions and methods presented in recommender systems.
Fairness notions can be defined, assessed and integrated to the algorithms for different stakeholders. For each stakeholder here we investigate the prior work and categorize it based on previous fairness notions such as group fairness, individual fairness, etc.

        % consumer fairness
        % individual fairness
        % group fairness
        % fairness beyond accuracy
        % more complex scenarios
\subsubsection{Consumer Side Fairness}
% consumer side fairness
Consumer fairness or (C-fairness) is concerned with the fair treatment of consumers (of recommendations) and the impact that recommendations have specifically on marginalized groups. Such objectives are sometimes required by law.
    
\paragraph{Individual Fairness}
%individual fairness
In collaborative filtering, the recommendations are build based on the similarity of users. And the goal is to use these similarities to recommend items to users in a personalized way. In other words, the recommendations of similar users are similar, thus they are treated similarly. This property might appear similar to the definition of individual fairness, although the metric based on which the similarity is calculated is different. As an example,
collaborative filtering uses the rating behavior of users whereas individual fairness looks at the user demographic information to calculate similarity which is hard to get by.
Another issue is that while similar users will be treated similarly, but since the data is not rich for minority groups, statistically, all of the users in that group might be treated unfairly when compared to the whole.
As an example, if women in a job recommendation platform tend to click on lower paying jobs, and since their rating behavior is similar, all of them get equally low paying jobs. Therefore, individual fairness and personalization might have similar goals, but the similarity metric and the information they use is different.


\paragraph{Group Fairness}
% group fairness
To integrate group fairness in recommendation algorithms, we can imagine the utility that the consumers receive from recommendations and whether it is distributed equally or whether all the individuals are benefiting equally from it. Similar to the group fairness definition in machine learning, here also we have both categories of (a) statistical or demographic parity and (b) performance parity. The first category focuses on sub-group representation while the latter focuses on subgroup loss (or gain). Since the performance parity is the statistical parity in performance metrics, it has been called statistical parity in previous research.

we measure the distribution of the genders of the authors of books in user rating profiles and recommendation lists produced from this data.

\cite{ekstrand2018exploring} demonstrates that the distribution of the authors' genders in user rating profiles and recommendation lists produced from this data are very different while the collaborative filtering algorithms propagate this bias. As a conclusion, to achieve demographic parity of authors' genders in the recommendations, we need to ensure that group (authors' genders) proportions in the recommendation sets should be similar to group proportions in input ratings. 
Performance parity in recommendation is calculated based on the effectiveness of recommendations and whether different subgroups are experiencing the same accuracy or error. Although not all of the definitions of error in classification can be applied to recommendation settings. Similar to \cite{kamishima2016model}, Yao and Huang \cite{yao_huang_fatml-2017} have designed different error-based fairness metrics for collaborative filtering such as value unfairness, over-representation, under-representation, etc. They have compared the discrepancy between the actual and predicted ratings for protected and unprotected groups or inconsistencies between the predicted ratings for these groups. 
\cite{ekstrand2018all} has performed an off-line top-N evaluation of several collaborative filtering algorithms and has compared the results of different user demographics based on their NDCG.
\cite{steck2018calibrated} introduces the concept of miscalibration in recommendations which has been used detected as consumer unfairness. Miscalibration happens when the item preferences of the users in their profile isn't covered in the recommendations they receive. \cite{Kun2020calib} has discusses that usually smaller or niche subgroups receive more miscalibrated recommendations compared to bigger or more popular subgroups.
% \cite{burke2018balanced} has also compared the statistical parity in precision for different demographic groups in consumers. 
    % provider fairness
        % provider utility
        % individual fairness
        % group fairness
        
\subsubsection{Provider Side Fairness}        
Provider-side fairness or (P-fairness) is concerned with treating the suppliers of information or items that are being recommended fairly. We can think of recommendation opportunities as a resource and the fair distribution of those opportunities among providers as provider fairness. 
Much of the research on diversity and decreasing popularity bias, contributes to provider fairness. Although, provider fairness derives from the goal of social justice and promotes the content from the underprivileged groups to provide more opportunities for them to be discovered. 
To achieve P-fairness, \cite{mehtora2018towards} tried to ensure that different sub-groups are similarly represented in the recommendations.
    
The utility defined in this context is mostly defined as exposure. A recommendation list is a short list that provides limited opportunities to expose items to consumers. According to the user attention pattern, the items that are on top of the list, receive more attention and this attention decreases as we go down in the list. Therefore, not all the positions in a list have equal utilities and only one item in the whole list can have the most valuable position and receive the most benefit \cite{diaz2020}. Therefore provider utility is calculated over all the recommendation lists delivered to all the users, while to calculate each consumers utility we only need to look at each consumer's recommendation and nothing more. Most of the metrics introduced for provider utility are based on NDCG \cite{biega2018equity}. %\todo{cite}

\paragraph{Individual Fairness}
%individual fairness
individual fairness for providers mean that similar items should receive similar utility from the recommender system. As an example, items that bring the same utility to a user (the user likes them both), are considered similar and should receive the same exposure in a recommendation list. \cite{biega2018equity} aggregates each item's attention and relevance over multiple rankings and assumes the providers are being treated fairly if the attention they have received from users are proportional to their relevance. The similarity of providers can be calculated in many other different ways and is an open research area.
    
\paragraph{Group Fairness}
%group fairness
Group fairness for providers is concerned with a fair treatment of different provider sub-groups. This goal should be aligned with the goal of personalization which considers users preferences so it doesn't recommend provider groups to users that don't like them. \cite{kamishima2018recommendation} proposes that the recommendations outcomes should be statistically independent of a sub-group's protected attribute in order to have group fairness. In this case, the probability that an item shows up in a recommendation list is independent of its sensitive attribute.
Another method is to ensure that there is a fair representation of providers in the recommendation lists. Unfairness in this case is when there is a big divergence between the distribution of the provider groups in the lists and the target distribution. \cite{yang2017measuring,das2019conceptual}. This divergence can be calculated using KL-divergence, difference in probabilities, odds ratio, etc. \cite{biega2018equity} calculates the provider groups fairness by aggregating expected exposure over multiple rankings. Fairness in this context happens when each provider group receives an appropriate level of exposure. \cite{beutel2019fairness} incorporates a fair construct (disparate mistreatment) in BPR\cite{rendlebpr2009} loss function. In this construct, a fair ranking happens when ranking a relevant item over an irrelevant item is independent of its group membership. Their pairwise fairness objective is defined in two way: once between groups and once within groups.
    
\subsubsection{Other stakeholders}     
% subject fairness
% other stakeholders
Besides the consumers and providers, there might be other stakeholders in the system whose fairness matters. In the Uber Eats scenario, besides considering fairness for users (who place orders) and restaurants (providers), we might want to increase fairness for drivers as well. For example, we might not want to overbook one driver, while other drivers stay in the queue to take delivery orders.
Subject fairness is another instance of this type, where subject is a stakeholder entity. And the fair goal might be having a fair distribution of the subjects of items being recommended. For example, in news platforms, to avoid polarization in the society, we might want to have a fair representation of different points of views, or giving a fair coverage to different topics and avoiding certain popular topics from monopolizing news feeds.
% Chen Karako
% diversity contributes to subject fairness

% It's worth noting that methods that intend to increase fairness for other stakeholders might indirectly contribute to fairness for other stakeholders.

% There is considerable research in the area of diversity-aware recommendation~\cite{Vargas:2011:RRN:2043932.2043955,adomavicius2012improving}. Essentially, these systems treat recommendation as a multi-objective optimization problem where the goal is to maintain a certain level of accuracy, while also ensuring that recommendation lists are diverse with respect to some representation of item content. These techniques can be re-purposed for P-fairness recommendation by treating the items from the protected group as a different class and then optimizing for diverse recommendations relative to this definition.

% Note, however, that this type of solution does not guarantee that any given item is recommended fairly, only that recommendation lists have the requisite level of diversity. This distinction is known as list diversity vs catalog coverage in the recommendation literature and as individual vs. group fairness in fairness-aware classification~\cite{fairness}. List diversity can be achieved by recommending the same ``diverse'' items to everyone, without necessarily providing a fair outcome for the whole set of providers. In this work, we are using metrics that measure group fairness, but we will extend these results to individual fairness measures in future work.

\subsubsection{Dynamic Fairness}  
% fairness over time
To define, measure and incorporate fairness definitions in algorithms, we should take into account many different aspects of fairness as we mentioned above. Another important aspect to consider is measuring the dynamics of fairness. Recommendation engines, change over time, as they interact with their users, attract new users and lose other users in the process. Therefore achieving fairness in one iteration might not be enough and might overlook these temporal changes.
As an example, in these systems, feedback loops might occur and this phenomena causes the system to pay more attention to popular/dominant subgroups of users \cite{hashimoto2018fairness} and therefore lose their under-represented sub-groups (either in consumers or providers). \cite{zhang2019group} analyzes the dynamics of fairness in sequential decision-making and tries to achieve a more balance performance which improves user retention. \cite{Chaney2018} studies how recommendations become more and more homogeneous in iterative environments which leads to inequity of exposure among items. 
    
% our contributions

\subsection{Achieving Algorithmic Fairness}
To achieve algorithmic fairness, interventions can be made at different steps of the processing pipeline. \cite{Friedler2019} provides a broad overview of these methods.

\subsubsection{Pre-processing}
Pre-processing methods focus on compensating the existing biases in the dataset. \cite{chen2018why} suggest different data collection enhancements to compensate for the biases that occur due to data imbalance. In order to do so, if there is an under-represented group in the data, by collecting more data on that group or imputing the fundamental features of that group, we can see improvements in the performance metrics automatically. \cite{Feldman2015} modifies the numerical attributes in the data to equalize their marginal distributions conditioned on the sensitive attribute. In this way, these distributions will be independent of the sensitive attribute and therefore the outcome of the machine learning models which are trained on this data will be independent of the group membership. \cite{hajian2012methodology} suggests modifying the values of the attributes and labels in the data such that unfair association rules cannot be mined from the dataset. Some other approaches create intermediary (lower-dimensional) representations of the data points so as to hide the information about sensitive attributes, while keeping the utility
of the modified data for the required task \cite{zemel2013learning,lahoti2019ifair}. 
Despite all this work, except a few exceptions \cite{ekstrand2018all}, there isn't much work in the recommenders systems field that focuses on de-biasing the data in recommender systems. However, many of these work in machine learning are applicable to the data that is appropriate for recommender systems. 
I recognize the existing gap in this topic in the field, however, my current work doesn't contribute to this section.


\subsubsection{In-processing}
%in-processing methods
In-processing approaches try to improve the fairness of results by modifying the algorithms and by integrating fairness notions in their loss functions. Therefore the problem will turn into a multi-objective optimization problem that seeks to simultaneously maximize utility and fairness.
These types of algorithmic interventions, sometimes take the form of regularizers to control certain structural properties of the model in the optimization functions. 
Regularizers are usually used to control the complexity of the model and to prevent the model from overfitting, although they can capture unfairness of the model as well. For example, \cite{zafar2017fairness} proposes to add fairness constraints on top of the accuracy constraints in the optimization objective of a classifier. Fairness regularization has been used in other classification and regression problems such as \cite{kamishima2012fairness,berk2017convex} and recommendation problems. For example, \cite{kamishima2018recommendation, kamishima-} proposes to add an independence term to the loss function that penalizes any correlations between the sensitive attribute and the predicted ratings. This term can also be added to achieve consumer-side fairness\cite{kamishima2017considerations}. They also propose multiple non-independence measures as well such as the difference in mean ratings between groups and the mutual information between the predicted ratings and the sensitive attributes. \cite{yao_huang_fatml-2017} also uses a regularization approach to minimize disparate rating predictions errors rather than recommendation errors. \cite{beutel2017data} adds a penalty term to their pairwise ranking loss function, to ensure that the difference between the ranking scores of the relevant and irrelevant items is uncorrelated with the relevant item's sensitive attribute. It is also possible to directly optimize a learning-to-rank such as \cite{diaz2020} that uses such method to achieve equal expected exposure.

In section 2, I present Balanced Neighborhood method as an in-processing algorithm with the goal of balancing between fairness and accuracy term. We realize unfair recommendation in neighborhood based models, could be a cause of having a neighborhood that is too homogeneous. To achieve fairness in this approach, we added a regularizer that ensures a diverse neighborhood for users. In this way, we prevent the algorithm to form unfairly homogeneous neighborhoods. Additionally, our goal is to reach a balance between fairness and accuracy.

Despite all the progress in this approach to improve fairness, the goal of accuracy and fairness can contradict sometimes. Since traditionally a lot of these algorithms' objective functions are designed to achieve accuracy, adding fairness notions as an extra constraint might prevent the objective functions to converge. Therefore, post-processing methods provide alternatives where in-processing modification of algorithms is not fruitful.


\subsubsection{Post-processing}
% post-processing methods
Post-processing approaches focus on modifying the outputs of the algorithms to satisfy a fairness criteria. In these methods, fairness constraints will not interfere with the goals of the objective function, rather they intervene after the output is produced. This approach can be applied both to classification and recommendation problems.
In \cite{fish2016confidence}, the proposed method tries to shift the boundaries of the already trained classifiers to achieve statistical parity with minimal accuracy loss. \cite{hardt2016equality} tries to balance the true positive rates of different groups by modifying the decision score thresholds of a trained classifier. \cite{kamiran2010discrimination} proposes a methodology that relabels the nodes of a decision tree classifier in order to ensure demographic parity.

Fairness can also be improved by re-ranking the output lists which were produced with the goal of achieving a high relevance. There are two main approaches of reranking: (a) those that treat the problem as a global optimization task and try to improve fairness with respect to the entire list of recommendations and (b) those methods that focus on the fairness of single lists one at a time.
An example of the first approach is \cite{surer2018multistakeholder} that proposes a constrained optimization-based method to enhance fairness (item exposure) for multiple provider groups, avoid unfairness towards under-represented groups and ensures a minimum degree of diversity for consumers. These methods are useful for occasions when recommendations are generated and cached in advance.
A more common approach is to re-rank individual lists as they are generated. Such approaches use methods like MMR (Maximal Marginal Relevance) \cite{carbonell1998use} or xQuAD \cite{santos2010explicit} that were presented in the information retrieval literature. These methods propose a greedy list expansion approach, where the re-reranked list is generated by adding new items to this list where it satisfies a fairness or diversity criteria. This approach also provides the benefit of controlling the balance between the accuracy and the fairness goal. 
\cite{modani2017fairness} use a re-rank approach to enhance provider exposure while preserving relevance. \cite{Geyik2019} uses a greedy approach to produce rankings of job-candidates that have a fair distribution of their demographic attributes, simultaneously optimizing for fairness and relevance. \cite{zehlike2017fa} uses A-star algorithm for reranking to achieve fairness in a ranked list at depth K. The goal here is to re-arrange the ranked lists to meet a fair distribution of items from different protected groups while keeping the quality of ranked lists as high as possible. 

Overall, re-ranking approaches offer a number of advantages. First, the trade-off between accuracy and fairness can be tuned without re-learning the recommendation model (as we have to do in in-processing approaches). Second, researchers have found that re-ranking can achieve better trade-offs versus accuracy with this type of model~\cite{abdollahpouri2019managing,liu2019personalized}. Due to this advantages we choose to use the latter method to increase fairness. In section 4, I present three re-ranking algorithms: Fairness-Aware re-Ranking, Personalized Fairness-Aware re-Ranking and Opportunistic Fairness-Aware re-Ranking. All of these methods are greedy approaches that focus on the fairness of single lists one at a time and they are based on information retrieval approaches that are intended to increase aggregate diversity.


\subsection{Summary of Contributions}

My work here has primarily focused on developing recommendation approaches in which fairness metrics are jointly optimized along with recommendation accuracy. I have structured the problem in a way that the balance between these two goals can be controlled and set using a hyper-parameter. However, my goal is to improve fairness, while preserving the accuracy as much as possible. Throughout the following work, I recognize all the stakeholders in a recommendation setting and their fairness concerns. Although I considered to improve fairness for both consumers and providers in the Fairness through Balanced Neighborhoods and have demonstrated the fairness improvements for both parties, in the rest of my work, the stakeholder of interest is the provider. All of the following work's fairness definitions are group-based definitions not individual-based fairness definitions. In the following methods, I have used performance parity metrics to assess fairness of the results where these metrics where both accuracy-based and exposure-based.

I present my five main contributions to the fairness-aware recommendation field. After a thorough literature review of fairness in machine learning, I present 
\begin{itemize}
    \item \textbf{(1) Fairness through Balanced Neighborhoods}:
    Here, I present an in-processing method with the goal of improving fairness while preserving as much accuracy as possible.
    By adding a regularizer to the objective function, for each user a diverse neighborhood is generated. These user neighborhoods play an important role in creating the recommendations of users. A diverse neighborhood for each user ensures a non-biased (or less biased) set of recommendations. Here our, goal is to reach a balance between fairness and accuracy. Additionally, I define the concept of multi-sided fairness and demonstrate improvements in fairness both for two sides of the recommendation setting: consumers and providers. 
    
    \item \textbf{(2) Fairness-Aware Recommendation Re-ranking (FAR) and personalized fairness-aware re-ranking methods (PFAR)}: 
    I present two greedy re-ranking approaches here which are both based on XQuAD (an information retrieval method to improve diversification). Both methods are designed to improve the fairness / accuracy tradeoff for the protected providers. 
    Other contributions of this project is the definition of a group-fairness metric for providers and the adaptation of fairness concepts to micro-finance systems. Therefore, the designed methods here are purposed for a loan recommendation scenario although they can be adapted to other contexts.
    
    \item \textbf{(3) Opportunistic Fairness-Aware Re-ranking (OFAiR)}:
    Here, I introduce the concept of opportunistic fairness. Users are not willing to experience diversity in every aspect of their recommendation. We detect the areas in which the users show willingness to see diversity and we consider them as opportunities to increase fairness without sacrificing much accuracy.
    Additionally, I use a greedy re-ranking approach based on Maximal Marginal Relevance or MMR (an information retrieval method to improve diversification), with the goal of improving the accuracy / fairness trade-off for multiple provider groups at the same time. This post-processing approach is one of the few approaches that defines multi-aspect fairness and designs a method to improve different fairness goals of various providers in a simultaneous way. As an example, improving the visibility of impoverished loan borrowers with respect to different aspects such as: region of the world, their demographic information, loan amount, the economic sector, etc.
    
    
    \item \textbf{(4) SCRUF framework}:
    Here, I propose a novel framework for recommender systems called \textit{Social Choice for Re-ranking Under Fairness} (SCRUF). This framework is appropriate for dynamic environments where multiple fairness concerns matter. In this method, we use group-based exposure-based fairness definitions for providers.
    
    \item \textbf{(5) Fair Librec-auto}: I present my contributions to \libauto{}, an open-source Python package providing a wrapper for the well-known LibRec which provides implementation of various recommendation algorithms. My contributions to this project were the implementation and addition of in-processing and post-processing fairness algorithms and fairness metrics.
    
\end{itemize}
% % \subsection{Fairness}
% \subsection{fairness in ML}
\subsection{fairness in ML and Recommendation}

\subsubsection{Personalization}

The dominant recommendation paradigm, collaborative filtering, uses user behavior as its input, ignoring user demographics and item attributes~\cite{koren2015advances}. However, this does not mean that fairness with respect to such attributes is irrelevant. Consider a recommender system suggesting job opportunities to job seekers. The operator of such a system might wish, for example, to ensure that male and female users with similar qualifications get recommendations of jobs with similar rank and salary. The system would therefore need to defend against biases in recommendation output, even biases that arise due to behavioral differences: for example, male users might be more likely to click optimistically on high-paying jobs.

Defeating such biases is difficult if we cannot assert a shared global preference ranking over items. Personal preference is the essence of recommendation especially in areas like music, books, and movies where individual taste is paramount. Even in the employment domain, some users might prefer a somewhat lower-paying job if it had other advantages: such as a shorter commute time, or better benefits. Thus, to achieve the policy goal of fair recommendation of jobs by salary, a site operator will have to go beyond a personalization-oriented approach, identify key outcome variables such as salary, and control the recommendation algorithm to make it sensitive to these outcomes for protected groups.

\subsubsection{Multiple stakeholders}

As the example of job recommendation makes clear, a recommender system is often in the position of facilitating a transaction between parties, such as job seeker and prospective employer. Fairness towards both parties may be important. For example, at the same time that a job recommender system is ensuring that male and female users to get recommendations with similar salary distributions, it might also need to ensure that jobs at minority-owned businesses are being recommended to the most desirable job candidates at the same rate as jobs at white-owned businesses.

A \textit{multistakeholder recommender system} is one in which the end user is not the only party whose interests are considered in generating recommendations~\cite{soappaper,abdollahpouri_recommender_2017}. This term acknowledges that recommender systems often serve multiple goals and therefore a purely user-centered approach is insufficient. Bilateral considerations, such as those in employment recommendation, were first studied in the category of \textit{reciprocal recommendation} where a recommendation must be acceptable to both parties in a transaction~\cite{akoglu_valuepick:_2010}. Other reciprocal recommendation domains include on-line dating~\cite{reciprocal}, peer-to-peer ``sharing economy'' recommendation (such as AirBnB, Uber and others), on-line advertising \cite{targetadvertisingbiding}, and scientific collaboration~\cite{lopes2010collaboration,tang2012cross}.

When recommendations must account for the needs of more than just the two transacting parties, we move beyond reciprocal recommendation to multistakeholder recommendation. Today's web economy hosts a profusion of multisided platforms, systems of commerce and exchange that bring together multiple parties in a marketplace, where the transacting individuals and the market itself all share in the transaction~\cite{evans_matchmakers:_2016}. These platforms must by design try to satisfy multiple stakeholders. Examples include LinkedIn, which brings together professionals, employers and recruiters; Etsy, which brings together shoppers and small-scale artisans; and Kiva.org, which brings together charitably-minded individuals with third-world entrepreneurs in need of capital.

\subsubsection{Stakeholder utility}

Different recommendation scenarios can be distinguished by differing configurations of interests among the stakeholders. We divide the stakeholders of a given recommender system into three categories: consumers $C$, providers $P$, and platform or system $S$. The consumers are those who receive the recommendations. They are the individuals whose choice or search problems bring them to the platform, and who expect recommendations to satisfy those needs. The providers are those entities that supply or otherwise stand behind the recommended objects, and gain from the consumer's choice.\footnote{In some recommendation scenarios, like on-line dating, the consumers and providers are same individuals.} The final category is the platform itself, which has created the recommender system in order to match consumers with providers and has some means of gaining benefit from successfully doing so. 

Recommendation in multistakeholder settings needs to be approached differently from user-focused environments. In particular, we have found that formalizing and computing stakeholder utilities is a productive way to design and evaluate recommendation algorithms. Ultimately, the system owner is the one whose utility should be maximized: if there is some outcome valued by the recommender system operator, it should be included in the calculation of system utility. 

The system inevitably has objectives that are a function of the utilities of the other stakeholders. Multisided platforms thrive when they can attract and retain critical masses of participants on all sides of the market. In our employment example, if a job seeker does not find the system's recommendations valuable, he or she may ignore this aspect of the system or may migrate to a competing platform. The same is true of providers; a company may choose other platforms on which to promote its job openings if a given site does not present its ads as recommendations or does not deliver acceptable candidates.

System utilities are highly domain-specific: tied to particular business models and types of transactions that they facilitate. If there is some monetary transaction facilitated by the platform, the system will usually get a share. The system will also have some utility associated with customer satisfaction, and some portion of that can be attributed to providing good recommendations. In domains subject to legal regulation, such as employment and housing, there will be value associated with compliance with anti-discrimination statutes. There may also be a (difficult to quantify) utility associated with an organization's social mission that may also value fair outcomes. All of these factors will govern how the platform values the different trade-offs associated with making recommendations.



\subsubsection{Multisided fairness}
Recommendation processes within multisided platforms can give rise to questions of multisided fairness. Namely, there may be fairness-related criteria at play on more than one side of a transaction, and therefore the transaction cannot be evaluated simply on the basis of the results that accrue to one side. There are three classes of systems, distinguished by the fairness issues that arise relative to these groups: consumers (C-fairness), providers (P-fairness), and both (CP-fairness).

\subsubsection{C-fairness}

A recommender system distinguished by C-fairness is one that must take into account the disparate impact of recommendation on protected classes of recommendation consumers. In the motivating example from~\cite{fairness}, a credit card company is recommending consumer credit offers. There are no producer-side fairness issues since the products are all coming from the same bank. 

Multistakeholder considerations do not arise in systems of this type. A number of designs could be proposed. One option that we explore in this paper is to design a recommender system following the approach of \cite{zemel2013learning} in generating fair classification. We generate neighborhoods for collaborative recommendations in such a way to have balanced representation of the opinions across groups. 

\subsubsection{P-fairness}

A system requiring P-fairness is one in which fairness needs to be preserved for the providers only. A good example of this kind of system is Kiva.org, an on-line micro-finance site. Kiva aggregates loan requests from field partners around the world who lend small amounts of money to entrepreneurs in their local communities. The loans are funded interest-free by Kiva's members, largely in the United States. Kiva does not currently offer a personalized recommendation function, but if it did, one can imagine a goal of the organization would be to preserve fair distribution of capital across its different regions in the face of well-known biases of users~\cite{lee2014fairness}. Consumers of the recommendations are essentially donors and do not receive any direct benefit from the system, so there are no fairness considerations on the consumer side. 

P-fairness may also be a consideration where there is interest in ensuring market diversity and avoiding monopoly domination. For example, in the on-line craft marketplace Etsy\footnote{www.etsy.com}, the system may wish to ensure that new entrants to the market get a reasonable share of recommendations even though they will have had fewer shoppers than established vendors. This type of fairness may not be mandated by law, but is rooted instead in the platform's business model.

There are complexities in P-fairness systems that do not arise in the C-fairness case. In particular, the producers in the P-fairness case are passive; they do not seek out recommendation opportunities but rather must wait for users to come to the system and request recommendations. Consider the employment case discussed above. We would like it to be the case that jobs at minority-owned businesses are recommended to highly-qualified candidates at the same rate that jobs at other types of businesses. The opportunity for a given minority-owned business to be recommended to an appropriate candidate may arrive only rarely and must be recognized as such. As with the C-fairness case, we will want to bound the loss of personalization that accompanies any promotion of protected providers. 

There is considerable research in the area of diversity-aware recommendation~\cite{Vargas:2011:RRN:2043932.2043955,adomavicius2012improving}. Essentially, these systems treat recommendation as a multi-objective optimization problem where the goal is to maintain a certain level of accuracy, while also ensuring that recommendation lists are diverse with respect to some representation of item content. These techniques can be re-purposed for P-fairness recommendation by treating the items from the protected group as a different class and then optimizing for diverse recommendations relative to this definition.

Note, however, that this type of solution does not guarantee that any given item is recommended fairly, only that recommendation lists have the requisite level of diversity. This distinction is known as list diversity vs catalog coverage in the recommendation literature and as individual vs. group fairness in fairness-aware classification~\cite{fairness}. List diversity can be achieved by recommending the same ``diverse'' items to everyone, without necessarily providing a fair outcome for the whole set of providers. In this work, we are using metrics that measure group fairness, but we will extend these results to individual fairness measures in future work.


% \section{fairness problems in recommendation}
% - IPM paper (probably skippabale since there's not much time to run all the experiments)
% the calibration paper with Nicole

\section{Regularization}
% balanced neighborhoods
In this section, we examine applications in which fairness with respect to consumers and to item providers is important. We focus on integrating a group fairness definition to the objective function of the well-known sparse linear method (SLIM) via adding a regularizer. And we show that variants of SLIM can be used to negotiate the tradeoff between fairness and accuracy.
% % \subsection{Balanced Neighborhoods}

% \noindent Bias and fairness in machine learning are topics of considerable recent research interest~\cite{pedreshi2008discrimination,fairness,bozdag_bias_2013}. A standard approach in this area is to identify a variable or variables representing membership in a protected class, for example, race in an employment context, and to develop algorithms that remove bias relative to this variable. See, for example, ~\cite{zemel2013learning,kamishima2012fairness,kamiran2010discrimination,zhang2017anti}.

% To extend this concept to recommender systems, we must recognize the key role of personalization. Inherent in the idea of recommendation is that the best items for one user may be different than those for another. It is also important to note that recommender systems exist to facilitate transactions. Thus, many recommendation applications involve multiple stakeholders and therefore may give rise to fairness issues for more than one group of participants~\cite{abdollahpouri_recommender_2017}.

% \subsubsection{Personalization}

% The dominant recommendation paradigm, collaborative filtering, uses user behavior as its input, ignoring user demographics and item attributes~\cite{koren2015advances}. However, this does not mean that fairness with respect to such attributes is irrelevant. Consider a recommender system suggesting job opportunities to job seekers. The operator of such a system might wish, for example, to ensure that male and female users with similar qualifications get recommendations of jobs with similar rank and salary. The system would therefore need to defend against biases in recommendation output, even biases that arise due to behavioral differences: for example, male users might be more likely to click optimistically on high-paying jobs.

% Defeating such biases is difficult if we cannot assert a shared global preference ranking over items. Personal preference is the essence of recommendation especially in areas like music, books, and movies where individual taste is paramount. Even in the employment domain, some users might prefer a somewhat lower-paying job if it had other advantages: such as a shorter commute time, or better benefits. Thus, to achieve the policy goal of fair recommendation of jobs by salary, a site operator will have to go beyond a personalization-oriented approach, identify key outcome variables such as salary, and control the recommendation algorithm to make it sensitive to these outcomes for protected groups.

% \subsubsection{Multiple stakeholders}

% As the example of job recommendation makes clear, a recommender system is often in the position of facilitating a transaction between parties, such as job seeker and prospective employer. Fairness towards both parties may be important. For example, at the same time that a job recommender system is ensuring that male and female users to get recommendations with similar salary distributions, it might also need to ensure that jobs at minority-owned businesses are being recommended to the most desirable job candidates at the same rate as jobs at white-owned businesses.

% A \textit{multistakeholder recommender system} is one in which the end user is not the only party whose interests are considered in generating recommendations~\cite{soappaper,abdollahpouri_recommender_2017}. This term acknowledges that recommender systems often serve multiple goals and therefore a purely user-centered approach is insufficient. Bilateral considerations, such as those in employment recommendation, were first studied in the category of \textit{reciprocal recommendation} where a recommendation must be acceptable to both parties in a transaction~\cite{akoglu_valuepick:_2010}. Other reciprocal recommendation domains include on-line dating~\cite{reciprocal}, peer-to-peer ``sharing economy'' recommendation (such as AirBnB, Uber and others), on-line advertising \cite{targetadvertisingbiding}, and scientific collaboration~\cite{lopes2010collaboration,tang2012cross}.

% When recommendations must account for the needs of more than just the two transacting parties, we move beyond reciprocal recommendation to multistakeholder recommendation. Today's web economy hosts a profusion of multisided platforms, systems of commerce and exchange that bring together multiple parties in a marketplace, where the transacting individuals and the market itself all share in the transaction~\cite{evans_matchmakers:_2016}. These platforms must by design try to satisfy multiple stakeholders. Examples include LinkedIn, which brings together professionals, employers and recruiters; Etsy, which brings together shoppers and small-scale artisans; and Kiva.org, which brings together charitably-minded individuals with third-world entrepreneurs in need of capital.

% \subsubsection{Stakeholder utility}

% Different recommendation scenarios can be distinguished by differing configurations of interests among the stakeholders. We divide the stakeholders of a given recommender system into three categories: consumers $C$, providers $P$, and platform or system $S$. The consumers are those who receive the recommendations. They are the individuals whose choice or search problems bring them to the platform, and who expect recommendations to satisfy those needs. The providers are those entities that supply or otherwise stand behind the recommended objects, and gain from the consumer's choice.\footnote{In some recommendation scenarios, like on-line dating, the consumers and providers are same individuals.} The final category is the platform itself, which has created the recommender system in order to match consumers with providers and has some means of gaining benefit from successfully doing so. 

% Recommendation in multistakeholder settings needs to be approached differently from user-focused environments. In particular, we have found that formalizing and computing stakeholder utilities is a productive way to design and evaluate recommendation algorithms. Ultimately, the system owner is the one whose utility should be maximized: if there is some outcome valued by the recommender system operator, it should be included in the calculation of system utility. 

% The system inevitably has objectives that are a function of the utilities of the other stakeholders. Multisided platforms thrive when they can attract and retain critical masses of participants on all sides of the market. In our employment example, if a job seeker does not find the system's recommendations valuable, he or she may ignore this aspect of the system or may migrate to a competing platform. The same is true of providers; a company may choose other platforms on which to promote its job openings if a given site does not present its ads as recommendations or does not deliver acceptable candidates.

% System utilities are highly domain-specific: tied to particular business models and types of transactions that they facilitate. If there is some monetary transaction facilitated by the platform, the system will usually get a share. The system will also have some utility associated with customer satisfaction, and some portion of that can be attributed to providing good recommendations. In domains subject to legal regulation, such as employment and housing, there will be value associated with compliance with anti-discrimination statutes. There may also be a (difficult to quantify) utility associated with an organization's social mission that may also value fair outcomes. All of these factors will govern how the platform values the different trade-offs associated with making recommendations.

% \subsection{Multisided fairness}

% Recommendation processes within multisided platforms can give rise to questions of multisided fairness. Namely, there may be fairness-related criteria at play on more than one side of a transaction, and therefore the transaction cannot be evaluated simply on the basis of the results that accrue to one side. There are three classes of systems, distinguished by the fairness issues that arise relative to these groups: consumers (C-fairness), providers (P-fairness), and both (CP-fairness).

% \subsubsection{C-fairness}

% A recommender system distinguished by C-fairness is one that must take into account the disparate impact of recommendation on protected classes of recommendation consumers. In the motivating example from~\cite{fairness}, a credit card company is recommending consumer credit offers. There are no producer-side fairness issues since the products are all coming from the same bank. 

% Multistakeholder considerations do not arise in systems of this type. A number of designs could be proposed. One option that we explore in this paper is to design a recommender system following the approach of \cite{zemel2013learning} in generating fair classification. We generate neighborhoods for collaborative recommendations in such a way to have balanced representation of the opinions across groups. 

% \subsubsection{P-fairness}

% A system requiring P-fairness is one in which fairness needs to be preserved for the providers only. A good example of this kind of system is Kiva.org, an on-line micro-finance site. Kiva aggregates loan requests from field partners around the world who lend small amounts of money to entrepreneurs in their local communities. The loans are funded interest-free by Kiva's members, largely in the United States. Kiva does not currently offer a personalized recommendation function, but if it did, one can imagine a goal of the organization would be to preserve fair distribution of capital across its different regions in the face of well-known biases of users~\cite{lee2014fairness}. Consumers of the recommendations are essentially donors and do not receive any direct benefit from the system, so there are no fairness considerations on the consumer side. 

% P-fairness may also be a consideration where there is interest in ensuring market diversity and avoiding monopoly domination. For example, in the on-line craft marketplace Etsy\footnote{www.etsy.com}, the system may wish to ensure that new entrants to the market get a reasonable share of recommendations even though they will have had fewer shoppers than established vendors. This type of fairness may not be mandated by law, but is rooted instead in the platform's business model.

% There are complexities in P-fairness systems that do not arise in the C-fairness case. In particular, the producers in the P-fairness case are passive; they do not seek out recommendation opportunities but rather must wait for users to come to the system and request recommendations. Consider the employment case discussed above. We would like it to be the case that jobs at minority-owned businesses are recommended to highly-qualified candidates at the same rate that jobs at other types of businesses. The opportunity for a given minority-owned business to be recommended to an appropriate candidate may arrive only rarely and must be recognized as such. As with the C-fairness case, we will want to bound the loss of personalization that accompanies any promotion of protected providers. 

% There is considerable research in the area of diversity-aware recommendation~\cite{Vargas:2011:RRN:2043932.2043955,adomavicius2012improving}. Essentially, these systems treat recommendation as a multi-objective optimization problem where the goal is to maintain a certain level of accuracy, while also ensuring that recommendation lists are diverse with respect to some representation of item content. These techniques can be re-purposed for P-fairness recommendation by treating the items from the protected group as a different class and then optimizing for diverse recommendations relative to this definition.

% Note, however, that this type of solution does not guarantee that any given item is recommended fairly, only that recommendation lists have the requisite level of diversity. This distinction is known as list diversity vs catalog coverage in the recommendation literature and as individual vs. group fairness in fairness-aware classification~\cite{fairness}. List diversity can be achieved by recommending the same ``diverse'' items to everyone, without necessarily providing a fair outcome for the whole set of providers. In this work, we are using metrics that measure group fairness, but we will extend these results to individual fairness measures in future work.

\subsection{Balanced Neighborhoods in Recommendation}

In~\cite{zemel2013learning}, the authors impose a fairness constraint on a classification by creating a \textit{fair representation}, a set of prototypes to which instances are mapped. The prototypes each have an equal representations of users in the protected and unprotected class so that the association between an instance and a prototype carries no information about the protected attribute. 

As noted above, the requirement for personalization in recommendation means that we have as many classification tasks as we have users. A direct application of the fair prototype idea would aggregate many users together and produce the same recommendations for all, greatly reducing the level of personalization and the recommendation accuracy. This idea must be adapted to apply to recommendation.

One of the fundamental ideas of collaborative recommendation is that of the \textit{peer user}, a neighbor whose patterns of interest match those of the target user and whose ratings can be extrapolated to make recommendations for the target user. One place where bias may creep into collaborative recommendation may be through the formation of peer neighborhoods. 

Consider the situation in Figure~\ref{fig:neighbor}. The target user here is the solid square, a member of the protected class. The top of the figure shows a neighborhood for this user in which recommendation will be generated only from other square users, that is, other protected individuals. We can think of this as a kind of segregation of the recommendation space. If the peer neighborhoods have this kind of structure relative to the protected class, then this group of users will only get recommendations based on the behavior and experiences of users in their own group. For example, in the job recommendation example that was mentioned previously, women would only get recommendations of jobs that have interested other women applicants, potentially leading to very different recommendation experiences across genders. 

\begin{figure}[bh]
    \centering
    \includegraphics[width=1.5in]{imgs/bln/neighborhood.pdf}
    \caption{Unbalanced (top) and balanced (bottom) neighborhoods}
    \label{fig:neighbor}
\end{figure}

To enhance the degree of C-fairness in such a context, we introduce the notion of a \textit{balanced neighborhood}. A balanced neighborhood is one in which recommendations for all users are generated from neighborhoods that are balanced with respect to the protected and unprotected classes. This is shown in the bottom half of Figure~\ref{fig:neighbor}. The target has an equal number of peers inside and outside of the protected class. In the case of job recommendation discussed before, this would mean that female job seekers get recommendations from some female and some male peers.

There are a variety of ways that balanced neighborhoods might be formed. The simplest way would be to create neighborhoods for each user that balance accuracy against group membership. However, this would be highly computationally inefficient, requiring the solution of a separate optimization problem for each user. 

In this research, we explore an extension of the well-known Sparse Linear Method (SLIM)~\cite{ning2011slim}. SLIM is well-known as a state-of-the-art technology for collaborative recommendation. It is a generalization of item-based recommendation in which a regression coefficient is learned for each $\langle user, item \rangle$ pair. It can be slower to optimize than factorization-based methods, but for our purposes, it has the important benefit that the learned coefficients are readily interpretable with regard to group membership. Our extension of SLIM uses regularization to control the way different neighbors are weighted, with the goal of achieving balance between protected and non-protected neighbors for each user.

\subsubsection{\textbf{Sparse Linear Method}}
\hfill

SLIM learns $\langle user, item \rangle$ regression weights through optimization, minimizing a regularized loss function. Although this is not proposed in the original SLIM paper, it is possible to create a user-based version of SLIM (labeled SLIM-U in~\cite{zheng2014cslim}), which generalizes the user-based algorithm in the same way. 

Assume that there are $M$ users (a set $U$), $N$ items (a set $I$), and let us denote the associated 2-dimensional rating matrix by $R$. SLIM is designed for item ranking and therefore $R$ is typically binary. We will relax that requirement in this work, We use $u_i$ to denote user $i$ and $t_j$ to denote the item $j$. An entry, $r_{ij}$, in matrix $R$ represents the rating of $u_i$ on $t_j$.

SLIM-U predicts the ranking score $\hat{s}$ for a given user, item pair $\langle u_i, t_j \rangle$ as a weighted sum:

\begin{equation}
    \hat{s}_{ij} = \sum_{k \in U}{w_{ik}r_{kj}}, 
\end{equation}
where $w_{ii} = 0$ and $w_{ik} >= 0$.

Alternatively, this can be expressed as a matrix operation yielding the entire prediction matrix $\hat{S}$:    
\begin{equation}
\hat{S} = WR,
\end{equation}
where $W$ is an $M x M$ matrix of user-user weights. For efficiency, it is very important that this matrix be sparse.

The optimal weights for SLIM-U can be derived by solving the following minimization problem:

\begin{equation}
\mbox{min}_W \frac{1}{2}\left\Vert R - WR \right\Vert^2 + 
    \lambda_1 \left\Vert W \right\Vert^1 +
    \frac{\lambda_2}{2}\left\Vert W \right\Vert^2,   
\end{equation}
subject to $W > 0$  and $diag(W) = 0$.

The $\left\Vert W \right\Vert^2$ term represents the $\ell_2$ norm of the $W$ matrix and $\left\Vert W \right\Vert^1$ represents the $\ell_1$ norm. These regularization terms are present to constrain the optimization to prefer sparse sets of weights. Typically, coordinate descent is used for optimization. Refer to \cite{ning2011slim} for additional details. 

% \subsubsection{Neighborhood Balance}
\paragraph{\textbf{Neighborhood Balance}}

Recall that our aim in fair recommendation is to eliminate segregated recommendation neighborhoods where protected class users only receive recommendations from other users in the same class. Such neighborhoods would tend to magnify any biases present in the system. If users in the protected class only are recommended certain items, then they will be more likely to click on those items and thus increase the likelihood that the collaborative system will make these items the ones that others in the protected group see.

To reduce the probability that such neighborhoods will form, we use the SLIM-U formalization of the recommendation problem, but we add another regularization term to the loss function, which we call the \textit{neighborhood balance} term. To describe this term, we will enrich our notation further by indicating $U^+$ to be the subset of $U$ containing users in the protected class with the remaining users in the class $U^-$. Let $W_i^+$ be the set of weights for users in $U^+$ and $W_i^-$ be the corresponding set of weights for the non-protected class. Then the neighborhood balance term $b_i$ for a given user $i$ is the squared difference between the weights assigned to peers in the protected class versus the unprotected class.

\begin{equation}
    b_i = (\sum_{w^+ \in W_i^+}{w^+} - \sum_{w^- \in W_i^-}{w^-})^2
\end{equation}

A low value for the neighborhood balance term means that the user's predictions will be generated by weighting protected and unprotected users on a relatively equal basis.\footnote{Note that this is a class-blind optimization that tries to build balanced neighborhoods for both the protected and unprotected users. It is also possible to formulate the objective such that it only impacts the protected class and we leave this option for future work.}

Another way to express this idea is to create a vector $p$ of dimension $M$. If $u_i$ is in $U^+$, then $p_i = 1$; if $u_i$ is in $U^-$, then $p_i = -1$. Then, the sum expressed above can be rewritten as $b_i = (p^T \cdot w_i)^2$. By adding up this term for all users and adding it to the loss function, we can allow the optimization process to derive weights with neighborhood balance in mind. This adapted version of SLIM-U we will call \textit{Balanced Neighborhood SLIM-U} or BN-SLIM-U.

As in the case of the original SLIM implementation, we can apply the method of coordinate descent to optimize the objective. The full loss function is as follows:

\begin{equation}
\begin{split}
 L = \frac{1}{2}\left\Vert R - WR \right\Vert^2 + 
    \lambda_1 \left\Vert W \right\Vert^1 + 
    \frac{\lambda_2}{2}\left\Vert W \right\Vert^2 + \\
    \frac{\lambda_3}{2}\sum_{i \in U}\left(\sum_{k \in U}p_iw_{ik}\right)^2,
\end{split}
\end{equation}
where $w_{ii}=0$ and $w_{ik}>=0$ and where $\lambda_3$ is a parameter controlling the influence of the neighborhood balance calculation on the overall optimization

This loss function retains the property of the original SLIM algorithm in that the rows of the weight matrix are independent, and the weights in each row (those for each user) can be optimized independently. The algorithm chooses one $w_{ik}$ weight and solves the optimization problem for that weight, repeating over all the weights until convergence is reached. If we take the derivative of $L$ with respect to a single weight $w_{ik}$, we obtain

\begin{equation}\label{eq:derivative}
\begin{split}
\frac{\partial L_i}{\partial w_{ik}} = \sum_{j \in I}{(r_{ij} - 
    \sum_{l \in U'}{w_{il}r_{lj}})} + w_{ik}\sum_{j \in I}{r_{kj}^2} +  \\
    \lambda_1 + \lambda_2w_{ik} + \lambda_3p_k\sum_{l \in U'}{p_lw_{il}} 
\end{split}
\end{equation}
where $U' = U - \{u_i, u_k\}$.

We then set this derivative to zero and solve for the value of $w_{ik}$ that produces this minimum. This becomes the coordinate descent update step. 

\begin{equation}\label{eq:update}
\begin{split}
    w_{ik} \leftarrow \frac{S\left(X_{ik}, \lambda_1\right)_+}
    {\sum_{j \in I}{r_{kj}^2} + \lambda_2 + \lambda_3} \\
    X_{ik} = \sum_{j \in I}{(r_{ij} - 
    \sum_{l \in U'}{w_{il}r_{lj}})}+\lambda_3p_k\sum_{l \in U'}{p_lw_{il}}
\end{split}
\end{equation}
where $S()_+$ is the soft threshold operator defined in ~\cite{friedman_pathwise_2007}.

\paragraph{\textbf{Item-based neighborhoods}}

As noted above, some applications may require P-fairness: making the recommendation outcomes fair relative to the items being recommended. In our micro-finance example, the operators of this site have the goal of providing equal exposure to loans from different geographic regions. To address the P-fairness case, we can use an analogous approach using item neighborhoods and item weights, ensuring that items in a protected group are in neighborhoods that have balanced membership of items from the unprotected group. The derivation of the loss function is exactly analogous, yielding another variant of the SLIM algorithm that we refer to as \textit{Balanced Neighborhood SLIM} or BN-SLIM.

\subsubsection{\textbf{Methodology}}
\hfill

In order to evaluate our balanced neighborhood approach, we conducted separate sets of experiments in both consumer- and provider-fairness. It is very difficult to find datasets that contain the kind of features that would be necessary to evaluate fairness-aware recommendation algorithms, especially related to user demographics in sensitive application areas such as employment. 

For the purposes of this paper, we are using the well-known MovieLens 1M dataset~\cite{movielens}, which contains gender information for each user, as well as ratings of 4,000 movies by 6,000 users. Movie recommendation is, of course, a domain of pure individual taste and therefore not an obvious candidate for fairness-aware recommendation. Following the example of \cite{yao2017beyond}, our approach to construct an artificial equity scenario within this data for expository purposes only, with the understanding that real scenarios can be approached with a similar methodology. 

Our consumer-fairness scenario centers on movie genres. It can be seen in this data that there is a minority of female users (1709 out of the total of 6040). Certain genres display a discrepancy in recommendation delivery to male and female users. For example, in the ``Crime'' genre, female users rate a very similar number of movies (average of 0.048\% of female profiles vs 0.049\% of male profiles) and rate them similarly: an average rating of 3.7 for both female and male users. However, our baseline unmodified SLIM-U algorithm recommends in the top 10 an average of 1.10 ``Crime'' movies per female user as opposed to 1.18 such movies to male users. We are still exploring the cause of this discrepancy, but it seems likely that there are influential female users with a lower opinion of this genre. 

Given that the rating profiles are similar but the recommendation outcomes are different, we can therefore conclude that the female users experience a deprivation of ``Crime'' movies compared to their male counter-parts. Similar losses can be observed for other genres. We are not asserting that there is any harm associated with this outcome. It is sufficient that these differences allow us to validate the properties of the BN-SLIM-U algorithm.

Our goal, then, is to reduce or eliminate genre discrepancies with minimal accuracy loss by constructing balanced neighborhoods for the MovieLens users. The $p$ vector in Equation~\ref{eq:update} therefore will have a 1 for female users and a -1 for male users. In the experiments below, we compare the user-based SLIM algorithm in its unmodified form and the balanced neighborhood version BN-SLIM-U.

In evaluating fairness of outcome, we use a variant of what is known in statistics as \textit{risk ratio} or \textit{relative risk} (RR)\cite{romei2014multidisciplinary}. We measure what is effectively \textit{relative opportunity}. In other words, we measure the observed probability of protected class items being recommended divided by the probability of unprotected class items being recommended. In our MovieLens experiments, we measure the number of movies in protected and unprotected genres included in recommendation lists  as the measure of outcome quality. We construct a consumer-side equity score, $E_c@k$ for recommendation lists of k items, as the ratio between the outcomes for the different groups. Let $P_i@k = {\rho_1, \rho_2, ..., \rho_k}$ be the top $k$ recommendation list for user $i$, and let $\gamma()$ be a function $\rho \rightarrow \{0,1\}$ that maps to 1 if the recommended movie is in a protected genre. Then:

\begin{equation}
E_c@k=\frac{\sum_{i \in U^+}{\sum_{\rho \in P_i@k}{\gamma(\rho)}}/|U^+|}
{\sum_{i \in U^-}{\sum_{\rho \in P_i@k}{\gamma(\rho)}}/|U^-|}
\end{equation}

$E_c@k$ will be less than 1 when the protected group is, on average, recommended fewer movies of the desired genre. It may be unrealistic to imagine that this value should approach 1: the metric does not correct for other factors that might influence this score -- for example, female users may rate a particular genre significantly lower and an equality of outcome should not be expected. While the absolute value of the metric may be difficult to interpret, it is still useful for comparing algorithms. The one with the higher $E_c@k$ is providing more movies in the given genre to the protected group. Note that this is an additive, utilitarian measure of outcome equity and does not take into account variations in user experience. More nuanced measures of distributional equity, including Pareto improvement, we leave for future work.

As in any multi-criteria setting, we must be concerned about any loss of accuracy that results from taking additional criteria into consideration. Therefore, we also evaluate the ranking accuracy of our algorithms in the results below. The measure that we use is normalized discounted cumulative gain (NDCG) measured at a specific list length. In this measure, an item appearing on a recommendation list accrues ``gain'' according to its position on the list -- thus the discount. The measure is normalized by comparing the algorithm's performance to the best ranking that could have been achieved. 

Let $P_i@10$ be a list of retrieved list of length 10 and let $\tau$ be an indicator function that is 1 for movies that the user liked and 0 for others. Then, DCG@10 is computed as

\begin{equation}
DCG@10 = \sum_{k=1}^{10}{\frac{\tau(\rho_k)}{log_2(k+1)}}
\end{equation}

NDCG@10 is this DCG@10 value divided by the optimal DCG, which occurs when all of the movies liked by the user and appearing the test set are ranked at the top of the list in their order of preference.

\paragraph{\textbf{Provider fairness}}

To evaluate our approach for provider fairness, we are using a dataset extracted from the Kiva.org microlending site using the site's API\footnote{http://build.kiva.org/}. Again, we have constructed our own scenario using this data, focusing on geographic region. In our dataset, we find that there are some geographic regions with a higher than average number of unfunded loans. In these regions, borrowers have a lower probability of getting the desired capital. See Table~\ref{tab:unfunded}.

\begin{table}
    \centering
\begin{tabular}{l|l|r}
    Category & Region & Unfunded \% \\ \hline
    Unprotected & North America & 1.73 \\
    & Eastern Europe & 0.99 \\
    & South America & 4.33 \\
    & Asia & 6.70 \\ \hline
    Protected & Africa & 10.57 \\
    & Middle East & 13.23 \\
    & Central America & 8.81 \\
\end{tabular}
    \caption{Percentage of unfunded loans by region}
    \label{tab:unfunded}
\end{table}

For the purposes of our experiments, we will assume that one of the goals of a microlending site is to equalize access to capital across geographic regions. Kiva does not currently offer personalized recommendation of loans to its users, but if it did, a fairness-aware recommendation approach could be used to promote the loans of borrowers in the underserved regions. 

We will therefore treat the under-represented regions collectively as the protected group and the other regions as the unprotected group. This enables us to use our item-based neighborhood balance algorithm described above. A more fine-grained approach to geographic equity that tries to balance across all regions would require additional algorithmic development and is left for future work. 

Again, we will represent fairness as a ratio of outcomes. It is simpler to compute in this case, as we are not dividing the recommendations by genre. The provider-side equity score, $E_p@k$, is defined on recommendation lists of k items. Let $L^+$ be the set of loans in the test set that are from the protected regions, and $L^-$ be the corresponding set from the unprotected regions. Also, let $\pi^+()$ be an indicator function $\rho \rightarrow \{0,1\}$ that maps to 1 if the recommended loan is from a protected region and $\pi^-$ is a similar function for the unprotected regions. Then:

\begin{equation}
E_p@k=\frac{\sum_{i \in U}{\sum_{\rho \in P_i@k}{\pi^+(\rho)}}/|L^+|}
{\sum_{i \in U}{\sum_{\rho \in P_i@k}{\pi^-(\rho)}}/|L^-|}
\end{equation}

$E_p@k$ will be less than 1 when loans from the protected regions are appearing less often on recommendation lists. As with $E_c$, this is a utilitarian measure, summing over all borrower regions, and does not speak to the the distribution across individual borrowers. Like $E_c$, it does not take the rank of recommended items into account.

\subsubsection{\textbf{Results}}
\hfill

We implemented the SLIM-U, BN-SLIM, and BN-SLIM-U algorithms using LibRec 2.0~\cite{guo2015librec}, and used its existing implementation of SLIM. We used 5-fold cross-validation as implemented within the library.

\paragraph{\textbf{Consumer fairness: MovieLens}}

Within the MovieLens 1M dataset, we selected the five genres on which the SLIM-U algorithm produced the lowest equity scores: ``Film-Noir'', ``Mystery'', ``Horror'', ``Documentary'', and ``Crime''. The parameters were set as follows: $\lambda_1 = 0.1$, $\lambda_2 = 0.001$, and (for BN-SLIM-U) $\lambda_3 = 25$\footnote{Because the balance term measures the difference in weights, it tends to be much smaller than the terms that measure the sums of weights. Therefore, the regularization constant must be much higher for the balance term to have an impact on the optimization.}. 

\begin{figure}[tbh]
    \centering
    \includegraphics[width=3.00in]{imgs/bln/genre-compare3.pdf}
    \caption{Equity score for SLIM-U and BN-SLIM-U. Line indicates equal percentage across genders}
    \label{fig:genre}
\end{figure}

Figure~\ref{fig:genre} shows the results of the experiment in terms of the equity scores for each genre. Perfect equity (1.0) is marked with the dashed line. As we can see, in every case, the balanced neighborhood algorithm produced an equity score closer to 1.0 than the unmodified algorithm. The largest jump is seen in the ``Horror'' genre, about 0.09 in the equity score or around 10\%.

In terms of accuracy, there was only a small loss of NDCG@10 between the two conditions. See Table~\ref{tab:ndcg}. The difference amounts to approximately 2\% loss in NDCG@10 for the balanced neighborhood version.

\begin{table}
\centering
\begin{tabular}{c|c}
    Algorithm &  NDCG@10 \\ \hline
    SLIM-U & 0.053 \\ \hline
    BN-SLIM & 0.052 \\ \hline
\end{tabular}
\caption{Ranking accuracy}
\label{tab:ndcg}
\end{table}

Because the balanced neighborhood algorithm is applied across all users, it also has the effect of showing male users movie genres that occur more frequently for female users. To see this effect, we examined the five genres with the highest $E_c@10$ values: ``Fantasy'', ``Animation'', ``War'', ``Romance'', and ``Western'' using the same parameter values as above. The results appear in Figure~\ref{fig:inverse-equity} and show a similar result. ``War'' is something of an anomaly here, both because it is perhaps unexpected to see it as a one of the more female-recommended genres and because the genre-balance algorithm pushes it to become more skewed rather than less. We are investigating the cause of this phenomenon. Overall, the BN-SLIM-U algorithm produces a recommendation experience in which the occurrence of gender-specific genres is more closely equalized, with small loss in ranking accuracy. 

\begin{figure}[bth]
    \centering
    \includegraphics[width=3in]{imgs/bln/inverse-genres3.pdf}
    \caption{Equity scores for female-preferred genres}
    \label{fig:inverse-equity}
\end{figure}

\paragraph{\textbf{Provider fairness: Kiva.org}}
Our dataset was extracted from Kiva's public API in September of 2016 and contains approximately 1 million loans funded by approximately 180,000 lenders. One challenge for collaborative recommendation in the microlending area is that loans are generally one-time endeavors. Unlike a movie that can be watched by an unrestricted number of viewers, a loan -- once funded -- disappears from Kiva.org and is not available for other lenders to view or support. Most loans are supported by from 1-330 lenders, by contrast, a popular movie in the MovieLens dataset might be rated by thousands of users. Thus, the lender-borrower relation is highly sparse, and loans have very small profiles.

To be able to apply the SLIM algorithm, we used a hybrid recommendation technique incorporating content data in the form of loan characteristics. We characterized each loan using five characteristics available from Kiva: borrower gender, borrower country, loan sector, loan purpose, and loan amount. Each of the original 1 million loan identifiers in the database was replaced with a psuedo-item identifier corresponding to the appropriate combination of loan characteristics. A 5-core transformation was then applied to the dataset, retaining only those users who had funded at least 5 psuedo-items and those psuedo-items with at least 5 funders. The retained dataset has 3,593 psuedo-items, 29,342 users and 393,035 ratings.

Kiva.org divides its borrowers into 9 geographic regions. As discussed above, for the purposes of this paper we are defining the protected group as those regions of the world where it appears to be more difficult to fund loans. (In Kiva.org, a loan that does not attract enough lenders over a 30 day period is marked as unfunded and dropped from the system.) As shown in Table~\ref{tab:unfunded}, the regions of North America, Eastern Europe, South America, and Asia have proportionately more funded loans than the regions of Africa, Middle East, and Central America\footnote{Our data set had only a single loan request from Australia.}. These regions where borrowers have lower funding percentages are treated as the protected group in our experiments.

With this transformation in place, it was possible to apply the SLIM algorithm and generate personalized recommendations. The regularization parameters were set as follows: $\lambda_1 = 0.01$ and $\lambda_2 = 0.001$. For BN-SLIM, $\lambda_3$ had a value of 0.9. Table~\ref{tab:results} shows the performance of the these algorithms in the provider fairness condition. Interestingly, the ranking accuracy, as measured by NDCG@10, actually increases between the conditions, indicating that the balanced neighborhood condition actually yields better recommendation lists than the unmodified SLIM algorithm. In addition, the $E_p@10$ value, which is unbalanced at 0.90 for SLIM is improved to close to 1.0, the equity target that we were aiming for.

\begin{table}
    \centering
\begin{tabular}{l|r|r}
    Algorithm & NDCG@10 & $E_p@10$ \\ \hline
    SLIM & 0.046 & 0.90 \\ \hline
    BN-SLIM & 0.049 & 1.05 \\ \hline
\end{tabular}
    \caption{Comparison of algorithm performance}
    \label{tab:results}
\end{table}


% \subsubsection{\textbf{Close Related Work}}
% There has been relatively little work on fairness in recommender systems.
% Most researchers in the area have defined fairness in terms of differing levels of accuracy for different classes of users. See, for example, \cite{DBLP:conf/recsys/KamishimaAAS14,kamisha-akaho-fatrec-2017,yao_huang_fatml-2017}. 

% As noted above, some special cases of provider-side fairness have been studied in the context of diversity-aware and long-tail recommendation. See, for example, \cite{Zhang:2008:AMI:1454008.1454030,adomavicius2012improving,o2004preserving,adomavicius2012improving}. 
% Our BN-SLIM algorithm can be seen as an approach to building systems that target particular diversity-aware recommendation problems, where the providers and/or items can be divided into two disjoint categories. However, the approach is particularly suited to fairness-aware contexts because the objective function is optimized precisely when the protected and unprotected groups are weighted the same by the algorithm. 

% The most obvious precursor for this research is the work of Dwork et al. in the area of fair representation~\cite{zemel2013learning,fairness}. The authors propose learning a mapping between the individual instances in the data to prototype instances with balanced membership such that protected group identities are not recoverable. Our application of this concept is different in that we are building on the standard nearest neighbor techniques in recommender systems and building balanced neighborhoods to ensure diversity among the peers from whom recommendations are generated. 

\subsubsection{\textbf{Conclusion and Future Work}}

% This paper extends ideas of fairness in classification to personalized recommendation. 
Our BN-SLIM algorithm can be seen as an approach to building systems that target particular diversity-aware recommendation problems, where the providers and/or items can be divided into two disjoint categories. However, the approach is particularly suited to fairness-aware contexts because the objective function is optimized precisely when the protected and unprotected groups are weighted the same by the algorithm. 

The most obvious precursor for this research is the work of Dwork et al. in the area of fair representation~\cite{zemel2013learning,fairness}. The authors propose learning a mapping between the individual instances in the data to prototype instances with balanced membership such that protected group identities are not recoverable. 

This paper extends this idea of fairness in classification to personalized recommendation. However, our application of this concept is different in that we are building on the standard nearest neighbor techniques in recommender systems and building balanced neighborhoods to ensure diversity among the peers from whom recommendations are generated. 

A key aspect of this extension is to note the tension between a personalized view of recommendation delivery and a regulatory view that values particular outcomes. The regulatory view is somewhat foreign to research in personalization, but there are strong arguments that total obedience to user preference is not always risk-free or desirable~\cite{pariser2011filter,sunstein2009republic}. This paper also introduces the concept of multisided fairness, relevant in multisided platforms that serve a matchmaking function. We identify consumer- and provider- fairness as properties desirable in certain applications and demonstrate that the concept of balanced neighborhoods in conjunction with the well-known sparse linear method can be used to balance personalization with fairness considerations.

In our future work, we plan to extend these findings in several ways. It is possible that a multisided platform may require fairness be considered for both consumers and providers at the same time: a CP-fairness condition. For example, a rental property recommender may treat minority applicants as a protected class and wish to ensure that they are recommended properties similar to unprotected renters. At the same time, the recommender may wish to treat minority landlords as a protected class and ensure that highly-qualified tenants are referred to them at the same rate as to landlords who are not in the protected class. One important question for future research is how the outcomes for each stakeholder and the overall system performance are affected by combining consumer- and provider-fairness concerns.
Another path to pursue is to have a more extensive experimentation of the fairness properties of the balanced neighborhood SLIM for both consumers and providers. We would like to test this idea on K-nearest neighbor method as well. Finally, we expect to publish a journal article of these thorough experiments in the Information and Management Journal.

% Another important area of research is to extend our measures of fairness. The additive measures used in this paper capture an aggregate representation of how recommendation results are changing for user and provider groups generally, but they do not permit fine-grained analysis of the tradeoffs experienced by individual users or providers. We do not know, for example, if the results of our Kiva.org experiments represent a Pareto improvement in system performance or just an average improvement over the stakeholder groups, and whether some subgroups are impacted more than others.

% One of the key challenges in this area is the domain-specificity of recommendation environments. The utilities that are delivered to each class of stakeholder are highly dependent on the type of item being recommended, the social function of the platform, and the interactions that it enables. It is therefore difficult to find appropriate data sets for experimentation and challenging to generalize across recommendation scenarios. 



\section{Re-ranking}

In this section, we focus on achieving provider fairness using a re-ranking approach.

The problem of promoting provider fairness while maintaining recommendation accuracy can be generally characterized as a multi-objective optimization problem. If optimal fairness and optimal recommendation accuracy could be achieved simultaneously, there would be no need for research in this area. However, optimizing recommendation accuracy often comes at the expense of provider fairness, due to various biases present in recommender systems, including popularity bias \cite{celma2008hits,lee2014fairness}, and user-base composition \cite{lin2019crank, yao2017beyond}. Research in provider fairness is therefore generally concerned with improving the tradeoff between fairness and accuracy, or in other words, increasing the amount of fairness that can be gained for a given degree of accuracy loss.

We motivate the problem in the context of loan recommendation where consumers are lenders and providers are borrowers. We propose two reranking methods: (1) Fairness-Aware Re-ranking (PFAR) and the personalized version of PFAR, and (2) Opportunistic Fairness-Aware Re-ranking (OFAiR).

%  try to increase the exposure of marginalized or protected borrowers. 
The re-ranking criterion can be regarded as modelling \textit{personalization} and \textit{fairness}, respectively, with a hyper-parameter $\lambda$ controlling the tradeoff between the two. We demonstrate that both methods achieve reasonable fairness / accuracy trade-offs and increase the exposure of the protected group(s) drastically. Although OFAiR achieves a better fairness / accuracy tradeoff compared to FAR/PFAR.

\input{03_far_pfar}
\input{04_ofair}

\section{Dynamic Fairness}
% SCRUF framework
\input{05_dynamic_fairness}

\section{Approach and Methodology}
% - librec-auto

Despite progress in recent years, reproducibility remains a challenge in recommender systems research \cite{beel2016towards}. Minor differences in parameters and experimental settings can yield incompatible results, which make it difficult to provide definitive answers about the relative properties of different algorithms. Progress in this area is supported by providing platforms on which comparative experiments can be conducted using declarative experimental configuration (so that experimental settings can be easily shared), with pre-implemented methodological workflows, and with a large library of algorithms for rapid benchmarking. 

In this section, I describe \libauto{}, an open-source command-line Python package providing a wrapper for the well-known LibRec 2.0 recommender systems algorithm library\footnote{www.librec.net}. My contributions to this project were the implementation of in-processing and post-processing fairness algorithms and fairness metrics.

\subsection{Librec-auto}
% In \cite{mansoury2018automating}, we introduced \libauto{}, an open-source command-line Python package providing a wrapper for the LibRec 2.0 recommender systems algorithm library\footnote{www.librec.net}. 
This tool has been presented in \cite{mansoury2018automating} and provides an environment that supports automated experimentation. Therefore it supports reproducibility of research in recommendation algorithms.

Key advantages of \libauto{} were its ability to support typical research workflows, to offer a declaration configuration system, and to supplement experiment execution with the scripted production of human-friendly outputs including visualizations.

We have now extended this platform in a number of ways, particularly to support research in fairness-aware recommendation. \libauto{} now supports the current 3.0 version of the LibRec library and can take advantage of the new algorithms (including deep learning algorithms) found there. The \libauto{} project has also enhanced LibRec with a suite of metrics for measuring the fairness of recommendation outcomes. Most significantly, the tool now supports recommendation re-ranking, a common approach to enhancing fairness, diversity, and other non-accuracy properties of recommendation outcomes.

\subsubsection{\textbf{Core features}}
\hfill

LibRec 3.0 is a Java-based recommendation generation platform. It has been available to the recommender systems community since 2015 \cite{guo2015librec}, and has large library of implemented recommendation algorithms (more than 70 as of this writing). The platform supports a variety of evaluation metrics and evaluation methodologies. However, our experience indicates that for practical experimentation and reproducibility research, LibRec by itself is not sufficient. For example, intermediate computational outputs, such as recommendation results, cannot be reused as input for new evaluation metrics, requiring the re-execution of potentially lengthy experiment executions.

We developed \libauto{}\footnote{github.com/that-recsys-lab/librec-auto} to retain the benefits of working with LibRec while adding support for experimentation. A sketch of the functionality of \libauto{} is provided in Figure~\ref{fig:librec-auto}. As the figure indicates, LibRec is encapsulated and its various component elements are used to execute particular portions of the experimental workflow. In addition, the optional re-ranking component allows for the study of re-ranking algorithms not within the scope of LibRec's design. The key inputs are data and an XML-based configuration file. A particular study may consist of multiple experiments, all of which are configured at the same time in the configuration file. Configuration files are modular, so that, for example, multiple studies can share the same methodology elements, preventing inadvertent misconfiguration. 

Although the figure indicates a straight-line of execution, parallelism is built into \libauto{} at the level of experiment execution. Because experiments can have lengthy execution times, the post-processing phase allows for integration with messaging platforms, including Slack, so that experimenters are notified when their tasks are complete. These messages can include visualizations of experimental output, to provide a quick overview of results. 

\begin{figure}
    \centering
    \includegraphics[width=5.25in]{imgs/la/librec-auto-diagram2.png}
    \caption{Schematic of experimentation workflow with \libauto{}. The LibRec library (Java, shown in grey) is encapsulated by \libauto{} (Python, shown in blue), which manages configuration, experimental outputs and post-processs. Added from \cite{mansoury2018automating} is the new re-ranking module shown in dark blue.}
    \label{fig:librec-auto}
    \vspace{-0.15in}
\end{figure}

\subsubsection{\textbf{Fairness-aware extensions}}
\hfill

Although \libauto{} has been under development since 2018, the latest release incorporates several key advances that specifically support common tasks in the study of recommendation fairness. These advances are (1) new evaluation metrics that report on fairness aspects of recommendation output, (2) an optional re-ranking step in the experiment pipeline, to support what is one of the most common category of fairness enhancing techniques, and (3) additional support for working with user (demographic) and item (content) features in algorithms and metrics. With these features, \libauto{} now can support a wide range of research activities in fairness-aware recommendation, and we will be adding additional capabilities in future releases.

Previously in the literature, many re-ranking algorithms have been proposed to achieve a balance between diversity and accuracy. The following methods try to achieve a fair representation between groups by penalizing the score of over-represented groups or reinforcing the score of the under-represented groups: (1) \textbf{FAR}, defined in \cite{liu2019farpfar}, combines a personalization-induced and fairness-induced scores with hyper-parameter $\lambda$; (2) \textbf{PFAR}, from \cite{liu2019farpfar}, adds a personalized weight to FAR, calculated based on item-features in user profile, representing the tolerance of the user for diverse resultsl and (3) \textbf{OFAiR} incorporates similar personalization and allows fine-grained control of protected group promotion when there are multiple protected groups \cite{sonboli2020opportunistic}. By contrast, (4) \textbf{FA*IR} \cite{zehlike2017fa} builds a queues of protected and unprotected items and draws from each queue to build the final re-ranked list. We also include two more general diversity-enhancing re-rankers first promoted in the information retrieval literature: (5) \textbf{MMR} diversifies result lists by greedily adding items with maximal marginal relevance \cite{carbonell1998use}, and (6) \textbf{XQuAD} defined in \cite{santos2010explicit} has similar goal to MMR algorithm, but it enhances diversity with respect to specific aspects. Finally, we include (7) \textbf{Calibrated Recommendations}, an algorithm closely tied to the Calibration metric above, which re-ranks recommendations to ensure a close match to the user's distribution of interests in item features~\cite{steck2018calibrated}. The re-ranking methods are part of \libauto{} and are implemented in Python.

Recommendation fairness and associated fairness metrics can be defined from the perspective of two main stakeholders: providers and consumers \cite{burke2017multisided}. Additionally, both provider-side and consumer-side metrics come in two basic varieties: exposure-based and hit-based. \textit{Exposure} metrics focus on the the appearance of protected items \cite{singh2018fairness} in a ranked list and \textit{hit-based} metrics take into account the suitability of the target user \cite{abdollahpouri2020multistakeholder}. Many metrics have been offered to measure recommendation fairness \cite{tsintzou2018bias,steck2018calibrated,beutel2019fairness,yao2017beyond,biega2018equity,castillo2019fairness,kuhlman2019fare,yang2017measuring}. We implement the following metrics in \libauto{} and where possible, both consumer-side and item-side versions of the metric are available: (1) \textbf{Discounted Proportional Fairness} (DPF), a hit-based fairness metric similar to the metric offered in \cite{castillo2019fairness} where it measures the ranking utility (nDCG) of the protected group with respect to the other groups. (2) \textbf{Calibration} \cite{steck2018calibrated}, a distribution-based metric that uses KL-Divergence to measure the difference in item category distribution between the preferences of users and their respective recommendation lists. (3) \textbf{Statistical parity}, based on the ideas discussed in \cite{zemel2013learning,ritov2017conditional}, measuring the difference in outcomes between protected and unprotected groups relative to various recommendation outcomes. Both ranking and prediction accuracy measures are supported. (4) \textbf{P-Percent-Rule} (PPR) discussed in \cite{biddle2006adverse}, is a two-sided extension of statistical parity \cite{barocas2016big}. (5) \textbf{Error-based} metrics proposed in Yao et al. \cite{yao2017beyond} including value-unfairness, absolute unfairness, underestimation unfairness, overestimation unfairness, and non-parity unfairness by Kamishima et al. \cite{kamishima2011fairness}. Additionally, we offer the following diversity-based metrics (6) \textbf{Intra-list distance (ILD)} \cite{ziegler2005improving}, a pairwise distance between all the item features in each user’s recommendation list, and (7) \textbf{Gini Index} calculated over the exposure of all the present groups in the recommendation list. All metrics are implemented in Java and integrated with the LibRec code base.

We plan future releases of \libauto{} to include integration with additional recommendation libraries, including LKPY~\cite{ekstrand2018lkpy}, LibFM~\cite{rendle2012factorization}, and DeepRec~\cite{zhang2019deeprec}.

\section{Proposed work and Timetable}
Overall, the presented methods in previous sections were looking to find a balance between accuracy and fairness either using a re-ranking method or a regularization based method. Most of the focus of the previous methods were on reaching provider-side fairness. 
% I intend to defend my thesis on December 2021. 
Here are the next steps for the projects that are introduced earlier and are going to be extended on my thesis by December 2021. 
\begin{itemize}
    \item \textbf{A Survey of Fairness in Recommender Systems}: Research on recommender systems fairness usually revolves around new algorithms. we are in the process of producing a journal article that surveys existing algorithms and compares their fairness properties on several data sets, from the perspective of both recommendation consumers and item providers. 

    We examine prominent examples from three different classes198of algorithms:  neighborhood-based, factorization using prediction loss, and factorization using ranking loss. From the neighborhood based algorithms we picked Sparse Linear Methods (SLIM) and item-based kNN. From the Matrix Factorization family using prediction loss we picked Biased Matrix Factorization, Weighted Regularized Matrix Factorization, Non-Negative Matrix Factorization, and from the factorization methods that use ranking loss we picked Bayesian Personalized Ranking. We are testing their fairness properties on the following datasets: The Movies Dataset, MovieLens 1M, Kiva dataset and Lastfm.
    
    I plan for these experiments and implementation to take about 2 months. This work is intended to be submitted as a journal article.
    
    \item \textbf{Fairness through Balanced Neighborhoods}: In this project, we built on the standard nearest neighbor techniques in recommender systems and built balanced neighborhoods to ensure diversity among the peers from whom recommendations are generated. In our future work for this project, we plan to extend these findings in several ways. We would like to have a more extensive experimentation of the fairness properties of the balanced neighborhood SLIM. And we would like to run these experimentation for both consumers and providers. We would also like to test this idea on different neighborhood-based methods besides SLIM. 
    
    It is possible that a multisided platform may require fairness be considered for both consumers and providers at the same time: a CP-fairness condition. For example, a rental property recommender may treat minority applicants as a protected class and wish to ensure that they are recommended properties similar to unprotected renters. At the same time, the recommender may wish to treat minority landlords as a protected class and ensure that highly-qualified tenants are referred to them at the same rate as to landlords who are not in the protected class. One important question for future research is how the outcomes for each stakeholder and the overall system performance are affected by combining consumer- and provider-fairness concerns.
    
    Finally, we expect to publish a journal article of these thorough experiments in the Information and Management Journal. I plan to spend 2 month on the implementation and experiments of this project.
    
    \item \textbf{Fairness in Dynamic Recommender Systems}: In this paper, we conceptualized algorithmic fairness and recommendation fairness, in particular, as a problem of \textit{social choice}. That is, we define the task of computing a recommendation as a problem of arbitrating among the preferences of different individual agents to arrive at a single outcome. For our purposes, the agents in question include the user and also multiple \textit{fairness concerns} that may be active within a particular organization.
    
    The most important consequence of framing fairness as a problem of social choice is that it highlights the multiplicity and diversity of fairness (and other stakeholder) concerns that might be relevant in a given application. This approach allows us to be agnostic to different definitions and metrics of fairness and does not impose any particular structure on stakeholder preferences. The other important consequence is that we can use the extensive body of research on fairness in this field.
    
    We build the SCRUFF framework for dynamic adaptation of recommendation fairness using social choice to arbitrate between different re-ranking methods. We defined a set of choice functions, ranging from a simple fixed lottery to an adaptation of the probabilistic serial mechanism, and demonstrate their performance on two data sets where multiple fairness concerns have been defined. However, we found relatively minor differences between the different lottery mechanisms, except that the Allocation mechanism, which takes user preferences over features into account, provides lower variance in fairness over time and therefore a more consistently fair output.
    
    In this regard, there are many aspects and variations to the experiments in this framework. For example, the definition of fairness can be measured through various metrics that I have not explored them completely. An appropriate metric such as Generalized Cross-entropy would help us better show the differences in performance of the various methods. Also, since our method works on top of a base recommendation, the choice of the recommendation algorithm may have a great impact on the final result. In my previous experiments I did not explore these methods thoroughly. Another extension for this work would be to add some constraint in the fairness objective to avoid increasing fairness for aspects that have already gained enough exposure. One approach for such constraints can be achieved through Hinge loss. Finally, in the current work fairness may not always be guaranteed in any given period of time. New objectives can be defined to overcome this problem and I am going to explore them as another extension.
    
    % Therefore, we intend to experiment with different aspects of this framework and test different aspects of it. Firstly, we intend to test different fairness definitions in this framework, test different base recommendation algorithms and re-ranking methods. The main reason behind that is that all these parts add extra restrictions to the framework and might hinder the results to change. Secondly, we didn't have an appropriate evaluation method to observe the results. We intend to use Generalized Cross Entropy for this purpose. Our results for fairness weren't bounded, therefore we intend to use Hinge Loss function to add a cap to the fairness values. Lastly, we intend to design a method that could guarantee certain fairness proportionality in the outputs.
    
    I plan for these experiments to take about two to three months. This Project is intended to be submitted to either the ACM Conference Series on Recommender Systems 2021 or the ACM conference on Fairness, Accountability, and Transparency 2021.

    
\end{itemize}

I plan to spend two month on writing the thesis and eventually defend it on December 2021.

% \subsection{Fairness Survey}
% Research on recommender systems fairness usually revolves around new algorithms. we are in the process of producing a journal article that surveys existing algorithms and compares their fairness properties on several data sets, from the perspective of both recommendation consumers and item providers. 

% We examine prominent examples from three different classes198of algorithms:  neighborhood-based, factorization using prediction loss, and factorization using ranking loss. From the neighborhood based algorithms we picked Sparse Linear Methods (SLIM) and item-based Knn. From the Matrix Factorization family using prediction loss we picked Biased Matrix Factorization, Weighted Regularized Matrix Factorization, Non-Negative Matrix Factorization, and from the factorization methods that use ranking loss we picked Bayesian Personalized Ranking.
% We are testing their fairness properties on the following datasets: The Movies Dataset, MovieLens 1M, Kiva dataset and Lastfm.


% \subsection{Balanced Neighborhoods}
% % This paper extends ideas of fairness in classification to personalized recommendation. 
% Our BN-SLIM algorithm can be seen as an approach to building systems that target particular diversity-aware recommendation problems, where the providers and/or items can be divided into two disjoint categories. However, the approach is particularly suited to fairness-aware contexts because the objective function is optimized precisely when the protected and unprotected groups are weighted the same by the algorithm. 

% The most obvious precursor for this research is the work of Dwork et al. in the area of fair representation~\cite{zemel2013learning,fairness}. The authors propose learning a mapping between the individual instances in the data to prototype instances with balanced membership such that protected group identities are not recoverable. 

% This paper extends this idea of fairness in classification to personalized recommendation. However, our application of this concept is different in that we are building on the standard nearest neighbor techniques in recommender systems and building balanced neighborhoods to ensure diversity among the peers from whom recommendations are generated. 

% A key aspect of this extension is to note the tension between a personalized view of recommendation delivery and a regulatory view that values particular outcomes. The regulatory view is somewhat foreign to research in personalization, but there are strong arguments that total obedience to user preference is not always risk-free or desirable~\cite{pariser2011filter,sunstein2009republic}. This paper also introduces the concept of multisided fairness, relevant in multisided platforms that serve a matchmaking function. We identify consumer- and provider- fairness as properties desirable in certain applications and demonstrate that the concept of balanced neighborhoods in conjunction with the well-known sparse linear method can be used to balance personalization with fairness considerations.

% In our future work, we plan to extend these findings in several ways. It is possible that a multisided platform may require fairness be considered for both consumers and providers at the same time: a CP-fairness condition. For example, a rental property recommender may treat minority applicants as a protected class and wish to ensure that they are recommended properties similar to unprotected renters. At the same time, the recommender may wish to treat minority landlords as a protected class and ensure that highly-qualified tenants are referred to them at the same rate as to landlords who are not in the protected class. One important question for future research is how the outcomes for each stakeholder and the overall system performance are affected by combining consumer- and provider-fairness concerns.

% Another path to pursue is to have a more extensive experimentation of the fairness properties of the balanced neighborhood SLIM for both consumers and providers. We would like to test this idea on K-nearest neighbor method as well. Finally, we expect to publish a journal article of these thorough experiments in the Information and Management Journal.

% Another important area of research is to extend our measures of fairness. The additive measures used in this paper capture an aggregate representation of how recommendation results are changing for user and provider groups generally, but they do not permit fine-grained analysis of the tradeoffs experienced by individual users or providers. We do not know, for example, if the results of our Kiva.org experiments represent a Pareto improvement in system performance or just an average improvement over the stakeholder groups, and whether some subgroups are impacted more than others.

% One of the key challenges in this area is the domain-specificity of recommendation environments. The utilities that are delivered to each class of stakeholder are highly dependent on the type of item being recommended, the social function of the platform, and the interactions that it enables. It is therefore difficult to find appropriate data sets for experimentation and challenging to generalize across recommendation scenarios. 


% expanding balanced neighborhood on the provider-side fairness

% \subsection{Fairness-Aware Re-ranking / Personalized Fairness-Aware Re-ranking}

% In this work, we proposed a personalized fairness-aware re-ranking algorithm for microlending that can balance accuracy and fairness. We increase the coverage rate of borrowers' regions for Kiva.org to achieve borrower-side fairness, and we show that our algorithm can do so with minimal loss in ranking accuracy. In addition, our algorithm includes lender-specific weights that can be used to personalize the degree of loan diversity.

% In the future, we will consider the position bias into the fairness-aware recommendation for microlending. As discussed in this paper, the recommendation for microlending is moved forward by considering the coverage rate of borrowers and the lenders' diversity tolerance. However, the top positions are generally more valuable than the bottom ones \cite{robertson1977probability}. We plan to make a further assumption that the chance of exposure for an item depends on its position in ranking. Thus, incorporating such position bias into the re-ranking criteria for microlending is promising.

% In the future, we will study a number of variants of our algorithms presented here. We plan to explore different methods for computing personalized diversity tolerance factors, especially to solve the cold-start problem in the current algorithm. We also plan to examine variants of the re-ranking algorithm to take into account the size of each provider group's inventory. 

% FAR/PFAR is extremely strict in its requirement that each possible provider group appears at least once at the top of the list. Therefore, another variant to consider is one that can adjust the accuracy/fairness tradeoff in a dynamic way as items are ranked, valuing accuracy more at the top of the list and provider-side fairness more at the bottom of the list.

% Finally, we note that, in real-world recommendation applications, managing the tradeoff between accuracy and coverage of provider groups is not a single-shot process. Rather it is an online process, where a current lack of coverage can be compensated for at a later time, and where results are evaluated temporally. This would require making the algorithm sensitive to historical patterns of coverage, rather than just the results obtained in the current list. We intend to explore this type of algorithm design and evaluation in our future work.

% This project currently is an on-going project and we intend to develop a method using probabilistic serial allocation and submit it to the ACM Conference Series on Recommender Systems 2021.

% \subsection{Opportunistic Multi-aspect Fairness through Personalized Re-ranking}
% The results of our experiments show that OFAiR works as intended. Its proportion-based MMR model provides a much better tradeoff between ranking accuracy and fairness for the protected-unprotected case than the FAR/PFAR models explored in prior work. In the datasets under study, we show that users' tolerance for diversity varies across features, which justifies our approach of differentiating users based on the opportunities they represent for enhancing provider-side fairness. 

% We show that the combination of personalized, feature-specific, weights together with weights identifying protected feature values is effective with the feature-specific tolerance helping maintain accuracy and the feature weight promoting protected group items. As we showed, our method can be applied across multiple protected groups at the same time and can ensure fairness with respect to system's designed fairness goal for each feature.

% One of the challenges in this work is the lack of proper datasets that have user features and these datasets are specifically lacking in domains where fairness matters. Due to this issue, we chose the Movies dataset to show the capabilities of our method.

% As our future work in this section, we intend to run a more thorough experimentation with weights of the weighted cosine similarity and capture the influence of these weights on the final results. We also intend to use different recommendation algorithms as the base recommendation.

% A more general method to use is the metric learning approach, that assumes different dimensions and assigns weights to these dimensions accordingly. This is useful as it automatically assigns weights to dimensions not manually.

% The other approach to explore is to use voting methods in the fair resource allocation literature in the computational social choice field. Specifically, the cake cutting problem, where we want to allocate cake slices fairly where we assume users have different preferences for different layers of the cake which is similar to our research problem here, where users have different preferences over different dimensions.

% The results of this experimentation is intended to be submitted to the ACM FAcct 2021 conference.

% journal

% In our next work, we intend to explore further the idea of ``opportunity'' in subgroup-fairness-aware recommendation. In particular, when recommendations are delivered over time, prior outcomes relative to different protected groups may dictate what opportunities should be most salient at any given moment. We intend to publish this work later this year in The ACM Series on Recommender Systems.


% \subsection{Dynamic Fairness}
% In this paper, we conceptualize algorithmic fairness and recommendation fairness, in particular, as a problem of \textit{social choice}. That is, we define the task of computing a recommendation as a problem of arbitrating among the preferences of different individual agents to arrive at a single outcome. For our purposes, the agents in question include the user and also multiple \textit{fairness concerns} that may be active within a particular organization. 

% The move to frame fairness as a problem of social choice has several important consequences. First, it highlights the multiplicity and diversity of fairness (and other stakeholder) concerns that might be relevant in a given application. This approach allows us to be agnostic to different definitions and metrics of fairness and does not impose any particular structure on stakeholder preferences.

% Second, we are able to make use of the large body of research in computational social choice, including the study of fairness, that has emerged in the past decades. 

% Building on these ideas, we demonstrate the SCRUFF framework for dynamic adaptation of recommendation fairness using social choice to arbitrate between different re-ranking methods. We define a set of choice functions, ranging from a simple fixed lottery to an adaptation of the probabilistic serial mechanism, and demonstrate their performance on two data sets where multiple fairness concerns have been defined. We found relatively minor differences between the different lottery mechanisms, except that the Allocation mechanism, which takes user preferences over features into account, provides lower variance in fairness over time and therefore a more consistently fair output.

% %%%%%%%%%%%%%%%%%%%%%%%%%%%%%

% What evaluation function fits the dynamic fairness environment
% What's regret in this environment
% Using hinge loss for the objective function to bound the fairness achievement etc.
% trying out different fairness definitions
% designing a method that guarantees certain properties in the re-ranked list



\bibliographystyle{ACM-Reference-Format}
\bibliography{myref.bib}
\end{document}
